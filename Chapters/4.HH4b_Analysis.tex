\section{Presenting the HH to 4b analysis, maybe mention cross sections and so on} \label{section: HH4b}

\subsection{Goal}

Signature and goal

\subsection{Presenting the cutflows for this analysis} \label{subsection:cutflows}
To train our SPANet model we are working with Monte-Carlo simulations. 

present tight and loose selections

explain 2b/4n

This dataset contains data collected in 2022, during the Run 3 and we require that the mean PNet of the leading 2 b-tag jets > 0.65, and the 3° and 4° PNet b-tag < 0.2605 (medium WP). By imposing this requirement, we know that in this dataset there are very little signal events, 

\subsection{Signature}

Goal: Parametr lambda of the Higgs potential, affects the trilinear Higgs coupling . Study HH production is the only way to directly probe the H boson self coupling 

Production modes: gluon fusion (add cross section and cite AN)/ VBF

H -> 4b bc this decaying has the highest BR

\subsection{Presenting the run 2 analysis method}

\subsubsection{Signal and control regions}
\subsection{New pairing method}

In order to improve the pairing done using the Run 2 method we want to use the attention-based neural network SPANet, whose architecture will be presented in section \ref{section: spanet architecture}. As we previously mentioned, we are interested in the decay of two Higgs Boson particles into 2 b quarks respectively. Therefore, in the final state we want at least 4 jets that we want to match to the original b quarks coming from the Higgs' decay. However, even if our process should have 4 jets in the final state, due to pile-up or other QCD processes, we can have more than 4 jets in the final state. In order to identify our signal, we can use btagging, presented in section \ref{Btagging}.


Nevertheless, because of the pile-up in the detector and other QCD processes, we can have more than 4 jets in the final state. However, as we are interested in jets originated by b quarks, these have particular signature and can be differentiated from jets originated by other quarks or gluons. This process is called b-tagging.

We will be working with Montecarlo simulations for our signal of interest. \textbf{Need more info on this}. Idk maybe also explain gen vs reco jets , but maybe this has to be done before.
