%\DeclareSIUnit\barn{b}

%!TEX root = ../thesis.tex
% Author: Izaak Neutelings (February 2023)
% Description: My common macros

% IWN: text
\newcommand\TODO[1]  {\textsf{({\color{blue!60!black!40}\textbf{TODO}: #1})}\xspace}
\newcommand\ADDREF   {\textsf{\color{blue!60!black!40}\textbf{\small[add Ref.]}}\xspace}
\newcommand\FIXME[1] {\textsf{(\textbf{FIXME}: #1)}}
\newcommand{\Left}   {\textbf{Left}}
\newcommand{\Middle} {\textbf{Middle}}
\newcommand{\Right}  {\textbf{Right}}
\newcommand{\Top}    {\textbf{Top}}
\newcommand{\Bottom} {\textbf{Bottom}}
\newcommand{\Fig}[1] {Fig.~\ref{#1}\xspace}
\newcommand{\Eq}[1]  {Eq.~\eqref{#1}\xspace}
\newcommand{\tabsubtitle}[1]{\rule{0pt}{13pt}\textbf{#1}} % subtitle in table
\newcommand{\phmin}  {\phantom{-}} % for aligning pos/neg. values in centered columns
\newcommand*{\myvec}[1]{\vec{\mkern0mu#1}} % correct misalignment in \vec
\newcommand*{\defeq}{\mathrel{\vcenter{\baselineskip0.5ex \lineskiplimit0pt
                     \hbox{\scriptsize.}\hbox{\scriptsize.}}}=} % definition :=

% IWN: Standard Model
\newcommand{\Lg}[1]  {\mathcal{L}_\text{#1}} % Lagrangian
\newcommand{\ff}[2]{#1_\text{#2}} % fermion field
\newcommand{\aff}[2]{\overline{#1}\vphantom{#1}_\text{#2}} % anti-fermion field
\newcommand{\U}[2][]{\text{U}(#2)\ifthenelse{\isempty{#1}}{}{_\text{#1}}} % U(N)
\newcommand{\SU}[2][]{\text{SU}(#2)\ifthenelse{\isempty{#1}}{}{_\text{#1}}} % SU(N)
\newcommand{\GSM}    {{\ensuremath{\SU[C]{3}\times\SU[L]{2}\times\U[Y]{1}}}\xspace} % SM group
\newcommand{\Ggauge} {{\ensuremath{G_\mathrm{gauge}^\mathrm{global}}}\xspace}
\newcommand{\YW}     {{\ensuremath{Y}}\xspace} % weak hypercharge %_\text{W}
\newcommand{\JW}     {\ensuremath{J_\mathrm{W}}\xspace} % charged weak current
\newcommand{\thetaW} {{\ensuremath{\theta_\mathrm{W}}}\xspace} % Weinberg mixing angle
\newcommand{\thetaC} {{\ensuremath{\theta_\mathrm{C}}}\xspace} % Cabibbo mixing angle
\newcommand{\VCKM}   {\ensuremath{V_\mathrm{CKM}}\xspace} % CKM matrix
\newcommand{\GF}     {\ensuremath{G_\mathrm{F}}\xspace} % Fermi constant
\newcommand{\MW}     {\ensuremath{M_{\PW}}\xspace} % W boson mass
\newcommand{\MZ}     {\ensuremath{M_{\PZ}}\xspace} % Z boson mass
\newcommand{\MH}     {\ensuremath{M_{\PH}}\xspace}
\newcommand{\db}[2]  {{\ensuremath{\def\arraystretch{1}\begin{pmatrix}#1\\#2\end{pmatrix}}}\xspace} % doublet
% \newcommand{\C}      {\ensuremath{{C}}} % charge conjugate
\newcommand{\hc}     {{\ensuremath{\text{h.c.}}}\xspace} % hermitian conjugate

% IWN: common particles / final states
\newcommand{\tauh}   {{\ensuremath{\PGt\kern-0.5pt_\text{h}}}\xspace} % hadronic tau
\providecommand{\tt} {} % create dummy to prevent "undefined" error in OverLeaf
\renewcommand{\tt}   {{\ensuremath{\PQt\PAQt}}\xspace}
%\renewcommand{\ll}   {{\ensuremath{\ell\ell'}}\xspace}
\newcommand{\ee}     {{\ensuremath{\Pe\Pe}}\xspace}
\newcommand{\mumu}   {{\ensuremath{\PGm\PGm}}\xspace}
\newcommand{\emu}    {{\ensuremath{\Pe\PGm}}\xspace}
\newcommand{\ltau}   {{\ensuremath{\ell\tauh}}\xspace}
\newcommand{\etau}   {{\ensuremath{\Pe\tauh}}\xspace}
\newcommand{\mutau}  {{\ensuremath{\PGm\tauh}}\xspace}
\newcommand{\tautau} {{\ensuremath{\PGt\kern-0.2pt\PGt}}\xspace}
\newcommand{\ditau}  {{\ensuremath{\tauh\kern-0.5pt\tauh}}\xspace}
\newcommand{\btau}   {{\ensuremath{\PQb\PGt}}\xspace}
\newcommand{\LQ}     {{\ensuremath{\mathrm{LQ}}}\xspace}
\newcommand{\ALQ}    {{\ensuremath{\overline{\mathrm{LQ}}}}\xspace}
\newcommand{\Zjets}  {{\ensuremath{\PZ\PNfix+\text{jets}}\xspace}}
\newcommand{\Wjets}  {{\ensuremath{\PW\PNfix+\text{jets}}\xspace}}
\newcommand{\jtf}    {{\ensuremath{j\to\tauh}\xspace}}
\newcommand{\ltf}    {{\ensuremath{\ell\to\tauh}\xspace}}

% IWN: common variables
\newcommand{\BF}     {{\mathcal{B}}} % branching fraction
\newcommand{\ST}     {{\ensuremath{S_\mathrm{T}}}\xspace} % scalar-sum pt
\newcommand{\muR}    {{\ensuremath{\mu_\text{R}}\xspace}} % renormalization scale
\newcommand{\muF}    {{\ensuremath{\mu_\text{F}}\xspace}} % factorization scale
\newcommand{\mvis}   {{\ensuremath{m_\text{vis}}}\xspace} % visible mass
\newcommand{\mll}    {{\ensuremath{m_{\ell\ell}}}\xspace}
\newcommand{\mtauh}  {\ensuremath{m_{\tauh}}\xspace} % tau_h mass
\newcommand{\mLQ}    {{\ensuremath{m_{\LQ}}}\xspace}
\newcommand{\Irel}   {\ensuremath{I_\text{rel}}} % relative isolation
\newcommand{\Njets}  {\ensuremath{N_\text{jets}}\xspace}

% IWN: Statistics
%\newcommand{\DNLL}{{\ensuremath{-2\Delta\kern-1.3pt\ln\kern-1.2pt\mathcal{L}}}\xspace}
\newcommand{\CL}{\ensuremath{\text{CL}}\xspace} % needs to be overridden to C.L. for APS. Look out for \CL.
%\newcommand{\CLs}{\ensuremath{\text{CL}_\text{s}}\xspace}
%\newcommand{\CLsb}{\ensuremath{\text{CL}_\text{s+b}}\xspace}

%!TEX root = ../thesis.tex
% Author: Izaak Neutelings (February 2023)
% Description: Mimicking common CMS TDR macros & particle pennames
% Sources:
%   https://gitlab.cern.ch/tdr/utils/-/blob/master/general/hepparticles.sty
%   https://gitlab.cern.ch/tdr/utils/-/blob/master/general/heppennames2.sty
%   https://gitlab.cern.ch/tdr/utils/-/blob/master/general/ptdr-definitions.sty

% CMS TDR: text
\newcommand{\etal}   {\mbox{et al.}\xspace} %et al. - no preceding comma
% \newcommand{\ie}     {\mbox{i.e.}\xspace}
% \newcommand{\eg}     {\mbox{e.g.}\xspace}
\newcommand{\etc}    {\mbox{etc.}\xspace}

% CMS TDR: common macros
\newcommand{\PNfix}   {\hspace{-.04em}}
\newcommand*{\DOI}[1]{\href{http://dx.doi.org/#1}{\texttt{doi:#1}}} % CHECK REFERENCE 10.1103 !!!
\providecommand{\NA}{\ensuremath{\text{---}}}
\providecommand{\cmsTable}[1]{\resizebox{\textwidth}{!}{#1}}

% CMS TDR: units
% \newcommand{\unit}[1]{{\ensuremath{\text{\,#1}}}\xspace}
\newcommand{\mus}    {{\ensuremath{\,\text{\textmu s}}}\xspace} %\upmu
\newcommand{\mum}    {{\ensuremath{\,\text{\textmu m}}}\xspace}
\newcommand{\cm}     {{\ensuremath{\,\text{cm}}}\xspace}
\newcommand{\MeV}    {{\ensuremath{\,\text{Me\hspace{-.08em}V}}}\xspace}
\newcommand{\GeV}    {{\ensuremath{\,\text{Ge\hspace{-.08em}V}}}\xspace}
\newcommand{\TeV}    {{\ensuremath{\,\text{Te\hspace{-.08em}V}}}\xspace}
\newcommand{\fbinv}  {{\mbox{\ensuremath{\,\text{fb}^\text{$-$1}}}}\xspace}

% CMS TDR: SOFTWARE PROGRAMS
\newcommand{\GEANTfour} {{\textsc{Geant4}}\xspace}
\newcommand{\GEANTthree} {{\textsc{geant3}}\xspace}
\newcommand{\GEANT} {{\textsc{geant}}\xspace}
\newcommand{\FASTJET} {{\textsc{FastJet}}\xspace}
\newcommand{\FEWZ} {{\textsc{fewz}}\xspace}
\newcommand{\Toppp} {\textsc{Top$++$}\xspace}
\newcommand{\HERWIG} {{\textsc{herwig}}\xspace}
\newcommand{\PYTHIA} {{\textsc{pythia}}\xspace}
\newcommand{\MADGRAPH} {\textsc{MadGraph}\xspace}
\newcommand{\aMCATNLO}{a\textsc{mc@nlo}\xspace}
\newcommand{\MCATNLO} {\textsc{mc@nlo}\xspace}
\newcommand{\MGvATNLO}{\MADGRAPH{}5\_a\MCATNLO}
\newcommand{\POWHEG} {{\textsc{powheg}}\xspace}
%\newcommand{\TAUOLA} {\textsc{tauola}\xspace}
\newcommand{\DeepTau} {{\textsc{DeepTau}}\xspace}
\newcommand{\DeepCSV} {{\textsc{DeepCSV}}\xspace}
%\newcommand{\Djet}{\ensuremath{D_\text{jet}}\xspace} % DeepJet
%\newcommand{\DeepTauVSe}{\texttt{DeepTau2017v2p1VSe}\xspace}
%\newcommand{\DeepTauVSmu}{\texttt{DeepTau2017v2p1VSmu}\xspace}
%\newcommand{\DeepTauVSjet}{\texttt{DeepTau2017v2p1VSjet}\xspace}

% CMS TDR PARTICLE PENNAMES
% https://gitlab.cern.ch/tdr/utils/-/blob/master/general/heppennames2.sty
% grep '\\PZ' */*.tex *.tex
% sed -e 's/\\PZ/\\PZ/g' -i '' *.tex */*.tex # macOS
% https://mirror.foobar.to/CTAN/macros/latex/contrib/was/upgreek.pdf
% https://ctan.org/pkg/fntguide

% UNSLANT small greek letters to make them look straight for particle names
% https://tex.stackexchange.com/questions/145926/upright-greek-font-fitting-to-computer-modern
% https://tex.stackexchange.com/questions/236915/adjust-custom-made-upright-greek-letters-when-used-in-subscripts
\usepackage{scalerel}
\newsavebox{\foobox}
\newcommand{\slantbox}[2][0]{\mbox{%
       \sbox{\foobox}{#2}%
       \hskip\wd\foobox
      \pdfsave
       \pdfsetmatrix{1 0 #1 1}%
       \llap{\usebox{\foobox}}%
       \pdfrestore
}}

\newcommand\unslant[2][-.25]{%
  \mkern1.2mu%
  \ThisStyle{\slantbox[#1]{$\SavedStyle#2$}}%
  \mkern-1.2mu%
}
\newcommand\unslantlong[2][-.25]{% % for long letter like mus
  \mkern-0.2mu%
  \ThisStyle{\slantbox[#1]{$\SavedStyle#2$}}%
  \mkern-0.8mu%
}

\newcommand{\upalpha}{\unslant\alpha}
\newcommand{\upgamma}{\unslant\gamma}
\newcommand{\upeta}{\unslant\eta}
\newcommand{\upmu}{\unslantlong\mu}
\newcommand{\upnu}{\unslant\nu}
\newcommand{\uppi}{\unslant\pi}
\newcommand{\uprho}{\unslant\rho}
\newcommand{\uptau}{\unslant\tau}
%\newcommand{\upphi}{\unslantlong\phi}
%\newcommand{\upchi}{\unslantlong\chi}
\newcommand{\uppsi}{\unslant\psi}
\newcommand{\upomega}{\unslant\omega}

% CMS TDR: PENNAMES
% https://gitlab.cern.ch/tdr/utils/-/blob/master/general/heppennames2.sty
\newcommand{\Pe}{{\ensuremath{\text{e}}}\xspace}
\newcommand{\Pl}{{\ensuremath{\ell}}\xspace}% electron
\newcommand{\PGm}{{\ensuremath{\upmu}}\xspace} % muon
\newcommand{\PGt}{{\ensuremath{\uptau}}\xspace}
\newcommand{\PGn}{{\ensuremath{\upnu}}\xspace}
\newcommand{\PGnl}{{\ensuremath{\upnu_{\ell}}}\xspace}
\newcommand{\PGne}{{\ensuremath{\upnu_{\Pe}}}\xspace}
\newcommand{\PGnGm}{{\ensuremath{\upnu_{\PGm\mkern-0.9mu}}}\xspace}
\newcommand{\PGnGt}{{\ensuremath{\upnu_{\PGt\mkern-0.9mu}}}\xspace}
\newcommand{\PAGnl}{{\ensuremath{\overline\upnu_{\ell}}}\xspace} % anti-neutrino
\newcommand{\PAGn}{{\ensuremath{\overline\upnu}}\xspace} % anti-neutrino
\newcommand{\PAGne}{{\ensuremath{\overline\upnu_{\Pe}}}\xspace}
\newcommand{\PAGnGm}{{\ensuremath{\overline\upnu_{\PGm\mkern-.9mu}}}\xspace}
\newcommand{\PAGnGt}{{\ensuremath{\overline\upnu_{\PGt\mkern-.9mu}}}\xspace}
\newcommand{\PQu}{{\ensuremath{\text{u}}}\xspace} % for u quark
\newcommand{\PQd}{{\ensuremath{\text{d}}}\xspace} % for d quark
\newcommand{\PQc}{{\ensuremath{\text{c}}}\xspace} % for c quark
\newcommand{\PQs}{{\ensuremath{\text{s}}}\xspace} % for s quark
\newcommand{\PQt}{{\ensuremath{\text{t}}}\xspace} % for t quark
\newcommand{\PQb}{{\ensuremath{\text{b}}}\xspace} % for b quark
\newcommand{\PAQu}{{\ensuremath{\bar{\text{u}}}}\xspace} % for u antiquark
\newcommand{\PAQd}{{\ensuremath{\bar{\text{d}}}}\xspace} % for d antiquark
\newcommand{\PAQc}{{\ensuremath{\bar{\text{c}}}}\xspace} % for c antiquark
\newcommand{\PAQs}{{\ensuremath{\bar{\text{s}}}}\xspace} % for s antiquark
\newcommand{\PAQt}{{\ensuremath{\bar{\text{t}}}}\xspace} % for t antiquark
\newcommand{\PAQb}{{\ensuremath{\bar{\text{b}}}}\xspace} % for b antiquark
\newcommand{\PQq}{{\ensuremath{q}}\xspace} % generic quark
\newcommand{\PAQq}{{\ensuremath{\bar{q}}}\xspace}
\newcommand{\PGg}{{\ensuremath{\gamma}}\xspace} % photon (gamma)
\newcommand{\Pg}{{\ensuremath{g}}\xspace} % generic gluon
\newcommand{\PW}{{\ensuremath{\text{W}}}\xspace} % W boson
%\newcommand{\PV}{{\ensuremath{\text{V}}}\xspace}
\newcommand{\PZ}{{\ensuremath{\text{Z}}}\xspace} % Z boson
\newcommand{\PH}{{\ensuremath{\text{H}}}\xspace} % Higgs boson
\newcommand{\Zg}{{\ensuremath{\PZ\kern-0.5pt/\kern-0.5pt\gamma^*}}\xspace} % photon / Z
\newcommand{\PGp}{\ensuremath{\uppi}\xspace} % pion
\newcommand{\PGr}{\ensuremath{\uprho}\xspace} % rho
\newcommand{\Pa}{\ensuremath{\text{a}}\xspace} % a_1
\newcommand{\PK}{\ensuremath{\mathrm{K}}\xspace} % kaon
\newcommand{\PAK}{\ensuremath{\overline{\mathrm{K}}}\xspace} % kaon
\newcommand{\Pp}{\ensuremath{\mathrm{p}}\xspace} % proton
\newcommand{\Pn}{\ensuremath{\mathrm{n}}\xspace} % neutron
\newcommand{\PD}{\ensuremath{\mathrm{D}}\xspace} % D meson
\newcommand{\PB}{\ensuremath{\mathrm{B}}\xspace} % B meson
\newcommand{\PAB}{\ensuremath{\overline{\mathrm{B}}}\xspace} % B meson
\newcommand{\PJGy}{\ensuremath{\mathrm{J}\mspace{-2mu}/\mspace{-2mu}\uppsi}\xspace} %  J/psi meson
\newcommand{\PGU}{\ensuremath{\Upsilon}\xspace} % Upsilon

% CMS TDR: common particle combinations
\newcommand{\bsln} {\ensuremath{\PQb\to\PQc\ell\PAGnl}\xspace} % b -> c ell nu
\newcommand{\bstn} {\ensuremath{\PQb\to\PQc\PGt\PGnGt}\xspace} % b -> s tau nu
\newcommand{\ttbar}{{\ensuremath{\PQt\PAQt}}\xspace}
\newcommand{\bbbar}{{\ensuremath{\PQb\PAQb}}\xspace}
\newcommand{\EE}   {{\ensuremath{\Pe^+\Pe^-}}\xspace}
\newcommand{\MM}   {{\ensuremath{\PGm^+\PGm^-}}\xspace}
\newcommand{\TT}   {{\ensuremath{\PGt^+\PGt^-}}\xspace}
\newcommand{\LL}   {{\ensuremath{\ell^+\ell^-}}\xspace}

% CMS TDR: common particle physics symbols
\newcommand{\lumi}{{\ensuremath{\mathcal{L}}}\xspace}
\newcommand{\DR}{{\ensuremath{\Delta R}}\xspace}
\newcommand{\ET}{\ensuremath{E_\text{T}}\xspace}
\newcommand{\kt}{\ensuremath{k_\text{T}}\xspace}
\newcommand{\pt}{\ensuremath{p_\text{T}}\xspace}
\newcommand{\Ht}     {\ensuremath{H_\text{T}}\xspace}
\newcommand{\ptvec}  {{\ensuremath{\myvec{p}_\text{T}}}\xspace}
\newcommand{\ptmiss} {\ensuremath{\pt^\text{miss}}\xspace}
\newcommand{\Htmiss} {\ensuremath{\Ht^\text{miss}}\xspace}
\newcommand{\ptvecmiss}{\ensuremath{{\myvec{p}}_{\mathrm{T}}^{\kern1pt\text{miss}}}\xspace}
\newcommand{\ptvecvis}[1][]{\ensuremath{{\myvec{p}}_{\mathrm{T}\ifthenelse{\isempty{#1}}{}{,#1}}^{\kern1pt\text{vis}}}\xspace}
\newcommand{\MET}    {{\ensuremath{\text{MET}}}\xspace}
\newcommand{\ETmiss} {{\ensuremath{E_{\mathrm{T}}^{\text{miss}}}}\xspace}
\newcommand{\PT}     {\pt}
\newcommand{\mT}     {{\ensuremath{m_\text{T}}}\xspace}

\newcommand{\WH}{{\ensuremath{\text{W}(\ell\PGn)\text{H}}}\xspace}
\newcommand{\ZH}{{\ensuremath{\text{Z}(\ell\ell)\text{H}}}\xspace}
\newcommand{\ZnuH}{{\ensuremath{\text{Z}(\PGn\PGn)\text{H}}}\xspace}
\newcommand{\ZZ}{{\ensuremath{\text{ZZ}}}\xspace}
\newcommand{\WZ}{{\ensuremath{\text{WZ}}}\xspace}
\newcommand{\WW}{{\ensuremath{\text{WW}}}\xspace}
\newcommand{\Znunu}{{\ensuremath{\text{Z}\to\PGn\PAGn}}\xspace}
\newcommand{\Zll}{{\ensuremath{\text{Z}\to\LL}}\xspace}
\newcommand{\Hbb}{{\ensuremath{\text{H}\to\bbbar}}\xspace}


\newcolumntype{M}[1]{>{\centering\arraybackslash}m{#1}}