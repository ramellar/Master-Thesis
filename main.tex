\documentclass[11pt]{article}
\usepackage{graphicx} % Required for inserting images
\usepackage[letterpaper,top=3cm,bottom=3cm,left=3cm,right=3cm,marginparwidth=1.75cm]{geometry}
\usepackage{amsmath}
\usepackage{chngcntr} % For numbering within chapters

\numberwithin{figure}{section}  % Number figures as Figure 1.1, 1.2, etc.
\numberwithin{table}{section}   % Number tables as Table 1.1, 1.2, etc.
\numberwithin{equation}{section} % Number equations as (1.1), (1.2), etc.

% Useful packages

\usepackage{array}
\usepackage[notrig]{physics}
\usepackage{amssymb}
\usepackage[colorlinks=true]{hyperref}
\usepackage[T1]{fontenc}
\usepackage[dvipsnames]{xcolor}
%\date{March 2023}
\usepackage[style=numeric,sorting=none]{biblatex}
\addbibresource{bibliography.bib}
\usepackage{graphicx,caption}
\usepackage{xspace}
\usepackage{subcaption}
\usepackage{mathtools}
\usepackage{textalpha}
\usepackage{titlesec}
\usepackage{fancyhdr}
\usepackage{enumitem}
\usepackage{float}
\usepackage{indentfirst}
\usepackage{ragged2e}
\usepackage{multirow}


\usepackage{biblatex} %Imports biblatex package
\addbibresource{bibliography.bib}

\hypersetup{
    colorlinks=true,
    linkcolor= Cerulean ,
    citecolor= SpringGreen,
}



\setcounter{secnumdepth}{4}

\titleformat{\paragraph}
{\normalfont\normalsize\bfseries}{\theparagraph}{1em}{}
\titlespacing*{\paragraph}
{0pt}{3.25ex plus 1ex minus .2ex}{1.5ex plus .2ex}

%\DeclareSIUnit\barn{b}

%!TEX root = ../thesis.tex
% Author: Izaak Neutelings (February 2023)
% Description: My common macros

% IWN: text
\newcommand\TODO[1]  {\textsf{({\color{blue!60!black!40}\textbf{TODO}: #1})}\xspace}
\newcommand\ADDREF   {\textsf{\color{blue!60!black!40}\textbf{\small[add Ref.]}}\xspace}
\newcommand\FIXME[1] {\textsf{(\textbf{FIXME}: #1)}}
\newcommand{\Left}   {\textbf{Left}}
\newcommand{\Middle} {\textbf{Middle}}
\newcommand{\Right}  {\textbf{Right}}
\newcommand{\Top}    {\textbf{Top}}
\newcommand{\Bottom} {\textbf{Bottom}}
\newcommand{\Fig}[1] {Fig.~\ref{#1}\xspace}
\newcommand{\Eq}[1]  {Eq.~\eqref{#1}\xspace}
\newcommand{\tabsubtitle}[1]{\rule{0pt}{13pt}\textbf{#1}} % subtitle in table
\newcommand{\phmin}  {\phantom{-}} % for aligning pos/neg. values in centered columns
\newcommand*{\myvec}[1]{\vec{\mkern0mu#1}} % correct misalignment in \vec
\newcommand*{\defeq}{\mathrel{\vcenter{\baselineskip0.5ex \lineskiplimit0pt
                     \hbox{\scriptsize.}\hbox{\scriptsize.}}}=} % definition :=

% IWN: Standard Model
\newcommand{\Lg}[1]  {\mathcal{L}_\text{#1}} % Lagrangian
\newcommand{\ff}[2]{#1_\text{#2}} % fermion field
\newcommand{\aff}[2]{\overline{#1}\vphantom{#1}_\text{#2}} % anti-fermion field
\newcommand{\U}[2][]{\text{U}(#2)\ifthenelse{\isempty{#1}}{}{_\text{#1}}} % U(N)
\newcommand{\SU}[2][]{\text{SU}(#2)\ifthenelse{\isempty{#1}}{}{_\text{#1}}} % SU(N)
\newcommand{\GSM}    {{\ensuremath{\SU[C]{3}\times\SU[L]{2}\times\U[Y]{1}}}\xspace} % SM group
\newcommand{\Ggauge} {{\ensuremath{G_\mathrm{gauge}^\mathrm{global}}}\xspace}
\newcommand{\YW}     {{\ensuremath{Y}}\xspace} % weak hypercharge %_\text{W}
\newcommand{\JW}     {\ensuremath{J_\mathrm{W}}\xspace} % charged weak current
\newcommand{\thetaW} {{\ensuremath{\theta_\mathrm{W}}}\xspace} % Weinberg mixing angle
\newcommand{\thetaC} {{\ensuremath{\theta_\mathrm{C}}}\xspace} % Cabibbo mixing angle
\newcommand{\VCKM}   {\ensuremath{V_\mathrm{CKM}}\xspace} % CKM matrix
\newcommand{\GF}     {\ensuremath{G_\mathrm{F}}\xspace} % Fermi constant
\newcommand{\MW}     {\ensuremath{M_{\PW}}\xspace} % W boson mass
\newcommand{\MZ}     {\ensuremath{M_{\PZ}}\xspace} % Z boson mass
\newcommand{\MH}     {\ensuremath{M_{\PH}}\xspace}
\newcommand{\db}[2]  {{\ensuremath{\def\arraystretch{1}\begin{pmatrix}#1\\#2\end{pmatrix}}}\xspace} % doublet
% \newcommand{\C}      {\ensuremath{{C}}} % charge conjugate
\newcommand{\hc}     {{\ensuremath{\text{h.c.}}}\xspace} % hermitian conjugate

% IWN: common particles / final states
\newcommand{\tauh}   {{\ensuremath{\PGt\kern-0.5pt_\text{h}}}\xspace} % hadronic tau
\providecommand{\tt} {} % create dummy to prevent "undefined" error in OverLeaf
\renewcommand{\tt}   {{\ensuremath{\PQt\PAQt}}\xspace}
%\renewcommand{\ll}   {{\ensuremath{\ell\ell'}}\xspace}
\newcommand{\ee}     {{\ensuremath{\Pe\Pe}}\xspace}
\newcommand{\mumu}   {{\ensuremath{\PGm\PGm}}\xspace}
\newcommand{\emu}    {{\ensuremath{\Pe\PGm}}\xspace}
\newcommand{\ltau}   {{\ensuremath{\ell\tauh}}\xspace}
\newcommand{\etau}   {{\ensuremath{\Pe\tauh}}\xspace}
\newcommand{\mutau}  {{\ensuremath{\PGm\tauh}}\xspace}
\newcommand{\tautau} {{\ensuremath{\PGt\kern-0.2pt\PGt}}\xspace}
\newcommand{\ditau}  {{\ensuremath{\tauh\kern-0.5pt\tauh}}\xspace}
\newcommand{\btau}   {{\ensuremath{\PQb\PGt}}\xspace}
\newcommand{\LQ}     {{\ensuremath{\mathrm{LQ}}}\xspace}
\newcommand{\ALQ}    {{\ensuremath{\overline{\mathrm{LQ}}}}\xspace}
\newcommand{\Zjets}  {{\ensuremath{\PZ\PNfix+\text{jets}}\xspace}}
\newcommand{\Wjets}  {{\ensuremath{\PW\PNfix+\text{jets}}\xspace}}
\newcommand{\jtf}    {{\ensuremath{j\to\tauh}\xspace}}
\newcommand{\ltf}    {{\ensuremath{\ell\to\tauh}\xspace}}

% IWN: common variables
\newcommand{\BF}     {{\mathcal{B}}} % branching fraction
\newcommand{\ST}     {{\ensuremath{S_\mathrm{T}}}\xspace} % scalar-sum pt
\newcommand{\muR}    {{\ensuremath{\mu_\text{R}}\xspace}} % renormalization scale
\newcommand{\muF}    {{\ensuremath{\mu_\text{F}}\xspace}} % factorization scale
\newcommand{\mvis}   {{\ensuremath{m_\text{vis}}}\xspace} % visible mass
\newcommand{\mll}    {{\ensuremath{m_{\ell\ell}}}\xspace}
\newcommand{\mtauh}  {\ensuremath{m_{\tauh}}\xspace} % tau_h mass
\newcommand{\mLQ}    {{\ensuremath{m_{\LQ}}}\xspace}
\newcommand{\Irel}   {\ensuremath{I_\text{rel}}} % relative isolation
\newcommand{\Njets}  {\ensuremath{N_\text{jets}}\xspace}

% IWN: Statistics
%\newcommand{\DNLL}{{\ensuremath{-2\Delta\kern-1.3pt\ln\kern-1.2pt\mathcal{L}}}\xspace}
\newcommand{\CL}{\ensuremath{\text{CL}}\xspace} % needs to be overridden to C.L. for APS. Look out for \CL.
%\newcommand{\CLs}{\ensuremath{\text{CL}_\text{s}}\xspace}
%\newcommand{\CLsb}{\ensuremath{\text{CL}_\text{s+b}}\xspace}

%!TEX root = ../thesis.tex
% Author: Izaak Neutelings (February 2023)
% Description: Mimicking common CMS TDR macros & particle pennames
% Sources:
%   https://gitlab.cern.ch/tdr/utils/-/blob/master/general/hepparticles.sty
%   https://gitlab.cern.ch/tdr/utils/-/blob/master/general/heppennames2.sty
%   https://gitlab.cern.ch/tdr/utils/-/blob/master/general/ptdr-definitions.sty

% CMS TDR: text
\newcommand{\etal}   {\mbox{et al.}\xspace} %et al. - no preceding comma
% \newcommand{\ie}     {\mbox{i.e.}\xspace}
% \newcommand{\eg}     {\mbox{e.g.}\xspace}
\newcommand{\etc}    {\mbox{etc.}\xspace}

% CMS TDR: common macros
\newcommand{\PNfix}   {\hspace{-.04em}}
\newcommand*{\DOI}[1]{\href{http://dx.doi.org/#1}{\texttt{doi:#1}}} % CHECK REFERENCE 10.1103 !!!
\providecommand{\NA}{\ensuremath{\text{---}}}
\providecommand{\cmsTable}[1]{\resizebox{\textwidth}{!}{#1}}

% CMS TDR: units
% \newcommand{\unit}[1]{{\ensuremath{\text{\,#1}}}\xspace}
\newcommand{\mus}    {{\ensuremath{\,\text{\textmu s}}}\xspace} %\upmu
\newcommand{\mum}    {{\ensuremath{\,\text{\textmu m}}}\xspace}
\newcommand{\cm}     {{\ensuremath{\,\text{cm}}}\xspace}
\newcommand{\MeV}    {{\ensuremath{\,\text{Me\hspace{-.08em}V}}}\xspace}
\newcommand{\GeV}    {{\ensuremath{\,\text{Ge\hspace{-.08em}V}}}\xspace}
\newcommand{\TeV}    {{\ensuremath{\,\text{Te\hspace{-.08em}V}}}\xspace}
\newcommand{\fbinv}  {{\mbox{\ensuremath{\,\text{fb}^\text{$-$1}}}}\xspace}

% CMS TDR: SOFTWARE PROGRAMS
\newcommand{\GEANTfour} {{\textsc{Geant4}}\xspace}
\newcommand{\GEANTthree} {{\textsc{geant3}}\xspace}
\newcommand{\GEANT} {{\textsc{geant}}\xspace}
\newcommand{\FASTJET} {{\textsc{FastJet}}\xspace}
\newcommand{\FEWZ} {{\textsc{fewz}}\xspace}
\newcommand{\Toppp} {\textsc{Top$++$}\xspace}
\newcommand{\HERWIG} {{\textsc{herwig}}\xspace}
\newcommand{\PYTHIA} {{\textsc{pythia}}\xspace}
\newcommand{\MADGRAPH} {\textsc{MadGraph}\xspace}
\newcommand{\aMCATNLO}{a\textsc{mc@nlo}\xspace}
\newcommand{\MCATNLO} {\textsc{mc@nlo}\xspace}
\newcommand{\MGvATNLO}{\MADGRAPH{}5\_a\MCATNLO}
\newcommand{\POWHEG} {{\textsc{powheg}}\xspace}
%\newcommand{\TAUOLA} {\textsc{tauola}\xspace}
\newcommand{\DeepTau} {{\textsc{DeepTau}}\xspace}
\newcommand{\DeepCSV} {{\textsc{DeepCSV}}\xspace}
%\newcommand{\Djet}{\ensuremath{D_\text{jet}}\xspace} % DeepJet
%\newcommand{\DeepTauVSe}{\texttt{DeepTau2017v2p1VSe}\xspace}
%\newcommand{\DeepTauVSmu}{\texttt{DeepTau2017v2p1VSmu}\xspace}
%\newcommand{\DeepTauVSjet}{\texttt{DeepTau2017v2p1VSjet}\xspace}

% CMS TDR PARTICLE PENNAMES
% https://gitlab.cern.ch/tdr/utils/-/blob/master/general/heppennames2.sty
% grep '\\PZ' */*.tex *.tex
% sed -e 's/\\PZ/\\PZ/g' -i '' *.tex */*.tex # macOS
% https://mirror.foobar.to/CTAN/macros/latex/contrib/was/upgreek.pdf
% https://ctan.org/pkg/fntguide

% UNSLANT small greek letters to make them look straight for particle names
% https://tex.stackexchange.com/questions/145926/upright-greek-font-fitting-to-computer-modern
% https://tex.stackexchange.com/questions/236915/adjust-custom-made-upright-greek-letters-when-used-in-subscripts
\usepackage{scalerel}
\newsavebox{\foobox}
\newcommand{\slantbox}[2][0]{\mbox{%
       \sbox{\foobox}{#2}%
       \hskip\wd\foobox
      \pdfsave
       \pdfsetmatrix{1 0 #1 1}%
       \llap{\usebox{\foobox}}%
       \pdfrestore
}}

\newcommand\unslant[2][-.25]{%
  \mkern1.2mu%
  \ThisStyle{\slantbox[#1]{$\SavedStyle#2$}}%
  \mkern-1.2mu%
}
\newcommand\unslantlong[2][-.25]{% % for long letter like mus
  \mkern-0.2mu%
  \ThisStyle{\slantbox[#1]{$\SavedStyle#2$}}%
  \mkern-0.8mu%
}

\newcommand{\upalpha}{\unslant\alpha}
\newcommand{\upgamma}{\unslant\gamma}
\newcommand{\upeta}{\unslant\eta}
\newcommand{\upmu}{\unslantlong\mu}
\newcommand{\upnu}{\unslant\nu}
\newcommand{\uppi}{\unslant\pi}
\newcommand{\uprho}{\unslant\rho}
\newcommand{\uptau}{\unslant\tau}
%\newcommand{\upphi}{\unslantlong\phi}
%\newcommand{\upchi}{\unslantlong\chi}
\newcommand{\uppsi}{\unslant\psi}
\newcommand{\upomega}{\unslant\omega}

% CMS TDR: PENNAMES
% https://gitlab.cern.ch/tdr/utils/-/blob/master/general/heppennames2.sty
\newcommand{\Pe}{{\ensuremath{\text{e}}}\xspace}
\newcommand{\Pl}{{\ensuremath{\ell}}\xspace}% electron
\newcommand{\PGm}{{\ensuremath{\upmu}}\xspace} % muon
\newcommand{\PGt}{{\ensuremath{\uptau}}\xspace}
\newcommand{\PGn}{{\ensuremath{\upnu}}\xspace}
\newcommand{\PGnl}{{\ensuremath{\upnu_{\ell}}}\xspace}
\newcommand{\PGne}{{\ensuremath{\upnu_{\Pe}}}\xspace}
\newcommand{\PGnGm}{{\ensuremath{\upnu_{\PGm\mkern-0.9mu}}}\xspace}
\newcommand{\PGnGt}{{\ensuremath{\upnu_{\PGt\mkern-0.9mu}}}\xspace}
\newcommand{\PAGnl}{{\ensuremath{\overline\upnu_{\ell}}}\xspace} % anti-neutrino
\newcommand{\PAGn}{{\ensuremath{\overline\upnu}}\xspace} % anti-neutrino
\newcommand{\PAGne}{{\ensuremath{\overline\upnu_{\Pe}}}\xspace}
\newcommand{\PAGnGm}{{\ensuremath{\overline\upnu_{\PGm\mkern-.9mu}}}\xspace}
\newcommand{\PAGnGt}{{\ensuremath{\overline\upnu_{\PGt\mkern-.9mu}}}\xspace}
\newcommand{\PQu}{{\ensuremath{\text{u}}}\xspace} % for u quark
\newcommand{\PQd}{{\ensuremath{\text{d}}}\xspace} % for d quark
\newcommand{\PQc}{{\ensuremath{\text{c}}}\xspace} % for c quark
\newcommand{\PQs}{{\ensuremath{\text{s}}}\xspace} % for s quark
\newcommand{\PQt}{{\ensuremath{\text{t}}}\xspace} % for t quark
\newcommand{\PQb}{{\ensuremath{\text{b}}}\xspace} % for b quark
\newcommand{\PAQu}{{\ensuremath{\bar{\text{u}}}}\xspace} % for u antiquark
\newcommand{\PAQd}{{\ensuremath{\bar{\text{d}}}}\xspace} % for d antiquark
\newcommand{\PAQc}{{\ensuremath{\bar{\text{c}}}}\xspace} % for c antiquark
\newcommand{\PAQs}{{\ensuremath{\bar{\text{s}}}}\xspace} % for s antiquark
\newcommand{\PAQt}{{\ensuremath{\bar{\text{t}}}}\xspace} % for t antiquark
\newcommand{\PAQb}{{\ensuremath{\bar{\text{b}}}}\xspace} % for b antiquark
\newcommand{\PQq}{{\ensuremath{q}}\xspace} % generic quark
\newcommand{\PAQq}{{\ensuremath{\bar{q}}}\xspace}
\newcommand{\PGg}{{\ensuremath{\gamma}}\xspace} % photon (gamma)
\newcommand{\Pg}{{\ensuremath{g}}\xspace} % generic gluon
\newcommand{\PW}{{\ensuremath{\text{W}}}\xspace} % W boson
%\newcommand{\PV}{{\ensuremath{\text{V}}}\xspace}
\newcommand{\PZ}{{\ensuremath{\text{Z}}}\xspace} % Z boson
\newcommand{\PH}{{\ensuremath{\text{H}}}\xspace} % Higgs boson
\newcommand{\Zg}{{\ensuremath{\PZ\kern-0.5pt/\kern-0.5pt\gamma^*}}\xspace} % photon / Z
\newcommand{\PGp}{\ensuremath{\uppi}\xspace} % pion
\newcommand{\PGr}{\ensuremath{\uprho}\xspace} % rho
\newcommand{\Pa}{\ensuremath{\text{a}}\xspace} % a_1
\newcommand{\PK}{\ensuremath{\mathrm{K}}\xspace} % kaon
\newcommand{\PAK}{\ensuremath{\overline{\mathrm{K}}}\xspace} % kaon
\newcommand{\Pp}{\ensuremath{\mathrm{p}}\xspace} % proton
\newcommand{\Pn}{\ensuremath{\mathrm{n}}\xspace} % neutron
\newcommand{\PD}{\ensuremath{\mathrm{D}}\xspace} % D meson
\newcommand{\PB}{\ensuremath{\mathrm{B}}\xspace} % B meson
\newcommand{\PAB}{\ensuremath{\overline{\mathrm{B}}}\xspace} % B meson
\newcommand{\PJGy}{\ensuremath{\mathrm{J}\mspace{-2mu}/\mspace{-2mu}\uppsi}\xspace} %  J/psi meson
\newcommand{\PGU}{\ensuremath{\Upsilon}\xspace} % Upsilon

% CMS TDR: common particle combinations
\newcommand{\bsln} {\ensuremath{\PQb\to\PQc\ell\PAGnl}\xspace} % b -> c ell nu
\newcommand{\bstn} {\ensuremath{\PQb\to\PQc\PGt\PGnGt}\xspace} % b -> s tau nu
\newcommand{\ttbar}{{\ensuremath{\PQt\PAQt}}\xspace}
\newcommand{\bbbar}{{\ensuremath{\PQb\PAQb}}\xspace}
\newcommand{\EE}   {{\ensuremath{\Pe^+\Pe^-}}\xspace}
\newcommand{\MM}   {{\ensuremath{\PGm^+\PGm^-}}\xspace}
\newcommand{\TT}   {{\ensuremath{\PGt^+\PGt^-}}\xspace}
\newcommand{\LL}   {{\ensuremath{\ell^+\ell^-}}\xspace}

% CMS TDR: common particle physics symbols
\newcommand{\lumi}{{\ensuremath{\mathcal{L}}}\xspace}
\newcommand{\DR}{{\ensuremath{\Delta R}}\xspace}
\newcommand{\ET}{\ensuremath{E_\text{T}}\xspace}
\newcommand{\kt}{\ensuremath{k_\text{T}}\xspace}
\newcommand{\pt}{\ensuremath{p_\text{T}}\xspace}
\newcommand{\Ht}     {\ensuremath{H_\text{T}}\xspace}
\newcommand{\ptvec}  {{\ensuremath{\myvec{p}_\text{T}}}\xspace}
\newcommand{\ptmiss} {\ensuremath{\pt^\text{miss}}\xspace}
\newcommand{\Htmiss} {\ensuremath{\Ht^\text{miss}}\xspace}
\newcommand{\ptvecmiss}{\ensuremath{{\myvec{p}}_{\mathrm{T}}^{\kern1pt\text{miss}}}\xspace}
\newcommand{\ptvecvis}[1][]{\ensuremath{{\myvec{p}}_{\mathrm{T}\ifthenelse{\isempty{#1}}{}{,#1}}^{\kern1pt\text{vis}}}\xspace}
\newcommand{\MET}    {{\ensuremath{\text{MET}}}\xspace}
\newcommand{\ETmiss} {{\ensuremath{E_{\mathrm{T}}^{\text{miss}}}}\xspace}
\newcommand{\PT}     {\pt}
\newcommand{\mT}     {{\ensuremath{m_\text{T}}}\xspace}

\newcommand{\WH}{{\ensuremath{\text{W}(\ell\PGn)\text{H}}}\xspace}
\newcommand{\ZH}{{\ensuremath{\text{Z}(\ell\ell)\text{H}}}\xspace}
\newcommand{\ZnuH}{{\ensuremath{\text{Z}(\PGn\PGn)\text{H}}}\xspace}
\newcommand{\ZZ}{{\ensuremath{\text{ZZ}}}\xspace}
\newcommand{\WZ}{{\ensuremath{\text{WZ}}}\xspace}
\newcommand{\WW}{{\ensuremath{\text{WW}}}\xspace}
\newcommand{\Znunu}{{\ensuremath{\text{Z}\to\PGn\PAGn}}\xspace}
\newcommand{\Zll}{{\ensuremath{\text{Z}\to\LL}}\xspace}
\newcommand{\Hbb}{{\ensuremath{\text{H}\to\bbbar}}\xspace}


\newcolumntype{M}[1]{>{\centering\arraybackslash}m{#1}}
\renewbibmacro{in:}{}
\renewbibmacro*{journal+issuetitle}{%
  \iffieldundef{journal}%
    {}% Do nothing if the journal is undefined
    {\printtext[journal]{%
      \printfield{journaltitle}}}%
  \newunit}

\renewbibmacro*{title}{%
  \ifboolexpr{
    test {\ifentrytype{book}}
    or
    test {\ifentrytype{mvbook}}
    or
    test {\ifentrytype{inbook}}
    or
    test {\ifentrytype{bookinbook}}
    or
    test {\ifentrytype{suppbook}}
  }
  {"\printfield[title]{title}"}  % Enclose title in double quotes
  {\printfield[title]{title}}  % Default for other types
}


\begin{document}

%\nocite{*}
\begin{titlepage}
\newcommand{\HRule}{\rule{\linewidth}{0.5mm}}
\begin{figure}[H]
\noindent
\begin{minipage}{.5\textwidth}
\begin{flushleft}
    \includegraphics[scale=0.15]{Logos/Unknown.png}
\end{flushleft}
\end{minipage}
\begin{minipage}{.5\textwidth}
\begin{flushright}
\includegraphics[scale=0.2]{Logos/logo_hep_centre_texte_coupe.png}
\end{flushright}
\end{minipage}

\end{figure}
\center
\HRule \\[0.4cm]
{ \huge \bfseries  Improving the reconstructed 4b signal final state pairing efficiency to Higgs boson candidate in the HH $\to$ 4b Run 3 analysis at CMS
 \\[0.15cm] }
 \HRule \\[0.5cm]
 \center 
\textsc{\LARGE\textbf{Master Thesis}}\\[1cm]
{\Large ETH Zurich}\\[0.5cm]
{\Large CMS}\\[1cm]

{\large\textbf{Supervisor: Alessandro Calandri, Mauro Donegà}}\\[1cm]
Author: Ruth AMELLA RANZ\\
M2 High Energy Physics
\\[1cm]
\textbf{Abstract}\\[1cm]
\justifying
The Higgs boson is the last building block of the Standard Model (SM) of particle physics. Its discovery in 2012 by the ATLAS \cite{ATLASdecouvhiggs} and CMS \cite{CMShiggsdecouv} experiments represents a major breakthrough in particle physics as it proves the existence of a scalar field, with which the particles in the SM interact to gain their mass. Nevertheless,  there is an ongoing experimental effort to further probe the properties of the Higgs boson, for instance, the self-coupling constant $\lambda$. This can be probed in di-Higgs (HH) production. In this thesis, the use of SPANet \cite{SPANet}, an attention-based neural network for jet paring and classification for signal extraction is proposed for the HH $\to$ 4b analysis. 

\end{titlepage}

\newpage
\pagenumbering{gobble}
\tableofcontents
\pagenumbering{arabic}
\pagestyle{fancy}
\fancyhf{}
\fancyhead[L]{\rightmark}
\fancyfoot[C]{\thepage}
\setlength{\headheight}{13.59999pt}
\addtolength{\topmargin}{-1.59999pt}
\setlength{\headheight}{25.22153pt}
\addtolength{\topmargin}{-11.62154pt}

\newpage

\section{Introduction} \label{intro}

The unification of the electromagnetic and weak forces leading to the Electroweak force was experimetally verified by the discovery of the Z boson in 1983 at the Super Proton Synchrotron (SPS) at CERN. Nevertheless, although in this theory the gauge bosons $W^\pm$ and Z are considered to be massless, the experimental results proved the opposite. To explain the mass of these particles, we need to introduce the Higgs field, as via the Brout–Englert–Higgs mechanism these gauge bosons can acquire mass.

Explain better the Higgs flied and the potential and so on. The Higgs boson is excitation of this field. It's discovery proves the existence of a fundamental scalar sector in the SM.

%Nowadays the Higgs boson mass measured by the CMS experiment is 125.53 GeV with a precision of 0.15 GeV (cite). To further characterize this particle, there is an ongoing experimental effort to determine the value of the self-coupling $\lambda$, which will also allow to probe the shape of the scalar potential of the Higgs mechanism.

\section{The LHC and the CMS detector} \label{section: CMS}

% Add LHC

The Large Hadron Collider (LHC) is currently the largest and highest-energy particle accelerator. It is a 27-km long circular hadron accelerator located 100 meters underground at CERN, near Geneva, on the border between France and Switzerland. Protons are accelerated and collide every 25 ns at the four interaction points, corresponding to the four main experiments: ATLAS \cite{ATLAS}, CMS \cite{CMS}, LHCb \cite{LHCb}, and ALICE \cite{ALICE}. Figure \ref{fig: LHC} shows the  LHC complex as well as the location of the four main experiences.

\begin{figure}[hbt]
    \centering
    \includegraphics[width=0.6\linewidth]{Images/3.CMS/Existing-CERN-accelerator-complex-with-Large-Hadron-Collider-LHC-Super-Proton.png}
    \caption{The Large Hadron Collider (LHC) complex \cite{LHCFIG}.}
    \label{fig: LHC}
\end{figure}

Ever since the LHC started taking data, there have been 3 periods of data taking each one referred to as "Run". Figure \ref{fig: Runs} shows the integrated luminosity of the three periods of data taking (Run 1, 2, and 3) as well as the center of mass energy of the collisions $\sqrt{s}$ in each period. The integrated luminosity \lumi is the integral of the luminosity with respect to time:

\begin{equation}
    \lumi= \int \frac{1}{\sigma} \frac{dN}{dt} dt
\end{equation}

\noindent where $\sigma$ is the cross-section and $N$ the number of events detected.

\begin{figure}[hbt]
    \centering
    \includegraphics[width=0.5\linewidth]{Images/3.CMS/Run plots.png}
    \caption{Integrated luminosity delivered by LHC as a function of data-taking years. The center of mass energy of the collisions $\sqrt{s}$ is also reported \cite{Runplot}.}
    \label{fig: Runs}
\end{figure}

%https://be-dep-op-lhc.web.cern.ch

%explain run 2 and 3

CMS (Compact Muon Solenoid) is one of the four detectors of the LHC. It is made of (starting from the center) a tracker, made of silicon sensors, that can measure the trajectory of the particles. Then, to measure the energy of the electrons and the photons, the electromagnetic calorimeter (ECAL) is used. The latter stops these particles completely via interactions (showers) with the PbWO$_4$ crystals. ECAL is surrounded by HCAL, the hadron calorimeter, which stops the hadrons and reconstructs their energy. This is followed by a large superconducting magnet, inducing a very strong magnetic field (3.81 T), that bends the trajectory of the particles. This allows to retrieve information in the tracker about the momentum and the sign of the charge of the particle due to the bending of its trajectory by the magnetic field. The last layer of the detector is the muon detector. Figure \ref{fig: CMS detector} shows the different layers of the CMS detector. 

\begin{figure}[hbt]
    \centering
    \includegraphics[width=0.8\linewidth]{Images/3.CMS/cms_160312_02.png}
    \caption{A three-dimensional view showing the different layers of the CMS detector \cite{CMS3D}.}
    \label{fig: CMS detector}
\end{figure}

Collisions in the detector take place every 25 ns, to retrieve the information from the collisions, CMS uses a multi-level trigger system. The CMS trigger system consists of:
\begin{itemize}
    \item Level 1 (L1) trigger: this trigger is implemented using FPGAs (Field Programmable Gate Array) and ASICs (Application-Specific Integrated Circuit) and allows for a fast readout of the detector with limited granularity. \cite{L1Trigger}
    \item High Level Trigger (HLT): the events that are accepted by the L1 trigger are passed to the HLT. This trigger is implemented as software algorithms that run on large clusters of commercial processors. \cite{HLT}
\end{itemize}

Figure \ref{fig: coord syst} shows the 3D coordinate system of the CMS detector. The accelerated proton beams are along the $z$-axis and the collisions occur at the interaction point (IP). The $xy$-plane as is observed in Figure \ref{fig: coord syst} is the transverse plane.

\begin{figure}[hbt]
    \centering
    \includegraphics[width=0.7\linewidth]{Images/4.HH4b Analysis/axis3D_CMS-005.png}
    \caption{Coordinate system of the CMS detector \cite{CMScoord}.}
    \label{fig: coord syst}
\end{figure}

\subsection{Kinematic variables used in the HH $\to$ 4b analysis} \label{subsection: kinematic vars}
The definitions of the observables that will be used in the following Sections are reported in this Section. Figure \ref{fig: coord syst} shows the CMS coordinate system.

\begin{itemize}
    \item The pseudorapidity $\eta$, defined as:
    \begin{equation}
        \eta=-\mathrm{ln}\bigg(\text{tan}(\frac{\theta}{2})\bigg)
    \end{equation}
    with $\theta$ being the angle between the beam axis ($z$) and the deviated particle or jet. When the deviation is low, the pseudo-rapidity either tends to $\pm \infty$ depending on whether it is along the $z$-axis or in the opposite direction (see Figure \ref{fig: coord syst}).
    \item $\Phi$ is the azimuthal angle in the $xy$ plane. Since this angle is in the transverse plane it is Lorentz-invariant.
\end{itemize}

Since the physics processes should not depend on the reference frame, the $\eta, \Phi$ is the metric used in the detector since $\Delta \eta$ and $\Phi$ is Lorentz-invariant. 

\begin{itemize}
    \item Transverse momentum of the particles \pt: projection of the momentum in the transverse plane, as shown Figure \ref{fig: coord syst}. In proton-proton collisions, the colliding partons carry a fraction of z-momentum that depends on the parton distribution functions (PDF), leading to a boost of the particles. To avoid this, the transverse momentum is used. 
    \item $\Delta R$ is the angular distance between two objects, for instance, jets. It is defined as:
    \begin{equation}
        \Delta R = \sqrt{(\Delta \eta)^2 + (\Delta \Phi)^2 }
    \end{equation}
    \item \Ht is the sum of all the \pt of the jets in a given reconstructed event.
    \item b-tagging is the identification (tagging) of jets containing the decay of a B hadron (b jets). The ParticleNet tagger (PNet), initially introduced in \cite{PNet},
    is used to identify the flavor of the jet (b-, c-, light-flavour jets or gluon jets). Thus, this network provides a score (b-tag score) which is correlated to the probability of a jet to originate from a b-quark, a c-quark or a light-flavor quark. The closer the score is to one, the more likely it is to be a jet originating from a b quark. Three working points (WPs) are defined:
    \begin{itemize}
        \item Tight WP: 0.1\% of misidentification (mistagging) of the flavor of the jet, i.e. only 0.1\% of the events above this point are jets coming from a quark with a lighter flavor or a gluon but are classified as b jets.
        \item Medium WP: 1\% of misidentification of the flavour of the jet
        \item Loose WP: 10\% of misidentification of the flavor of the jet
    \end{itemize}
    \item \pt regressed (\pt reg): by using the ParticleNet neural network it is possible to correct the raw jet \pt to the generator-level jet energy. To do so, two components are taken into account: the \pt is adjusted to be closer to the generator-level jet \pt, and the presence of neutrinos, not reconstructed in the detector, is also taken into account, by summing their transverse momentum to the jet.
    \item Impact parameter $d_0$ is the distance of closest approach between the daughter particle trajectory and the mother particle production point. The impact parameter (IP) is shown in Figure \ref{fig: IP}. The impact parameter in the $xy$ plane is noted $d_{xy}$ and along the z-axis $d_z$.
\end{itemize}

\begin{figure}[hbt]
    \centering
    \includegraphics[width=0.5\linewidth]{Images/3.CMS/impact parameter.png}
    \caption{Sketch showing the definition of the impact parameter (IP) as well as primary (PV) and secondary (SV) vertices in heavy-flavor jet production \cite{IP}. }
    \label{fig: IP}
\end{figure}


%Maybe nno need of too much on this part

% \subsection{Tracker}
% \subsection{Electromagnetic calorimeter (ECAL)}
% \subsection{Hadronic Calorimeter (HCAL)}
% \subsection{Solenoid}
% \subsection{Muon detector}
% \subsection{Triggers}
% \subsection{Event recontsruction}
% \subsubsection{Primary Vertex}
% \subsubsection{Pile-up}
% \subsubsection{Jet reconstruction}
% \subsubsection{Missing Energy}
% \subsubsection{B-tagging} \label{Btagging} b tag algorithms are used and the fact that we use th e btag fpr the morphing to 2b to 4b
% \subsubsection{pt reg} \label{subsubsec: btag-ptreg}
% The \pt regressed is the one computed using a DNN. The main idea is that, we train a DNN with information from the reco jets and we try to retrieve the original energy if these jets. For efficiency purposes this DNNN gives out the ratio of the predicted E/ original E. To improve the analysis, we use \pt reg, which means that we multiply the \pt measured by the detector times this ratio, which allows us ti have a more precise value of thee \pt of the objects we are using.
% \subsection{Monte-Carlo}
% \subsubsection{Gen/Reco jets}

% 0.1 perccent if misidentification if the flaviur
% 1
% 10

\section{Presenting the HH to 4b analysis} \label{section: HH4b}

\subsection{Aim of the HH to 4b analysis}

As mentioned in the Introduction (Section \ref{intro}) we introduce the the Brout–Englert–Higgs mechanism that allows to generate the masses of vector bosons without breaking the gauge invariance explicitly. This mechanism introduces the Higgs field $H$, a real scalar field with which massive particles interact to acquire mass. The Higgs boson is an excitation of this field, therefore its discovery by the ATLAS and CMS experiments proves the existence of a scalar field in the SM of particle physics. Nevertheless, to further probe this mechanism, we want to verify the shape of the Higgs potential postulated by the Higgs mechanism. As shown in Eq.(\ref{eq:Higgs potential}), the shape of this potential is governed by the parameter $\lambda$.

\begin{equation}
    V(\Phi^\dag \Phi)=-\mu^2 \Phi^\dag \Phi + \lambda (\Phi^\dag \Phi)^2
    \label{eq:Higgs potential}
\end{equation}

\noindent where $\Phi$, when parameterized in terms of a particular gauge, is the real scalar field that introduces the Higgs field:

\begin{equation}
    \Phi= \frac{1}{\sqrt{2}}\bigg( {0 \atop v + H} \bigg)
\end{equation}

\noindent where $H$ is the Higgs field and $v=\sqrt{\frac{\mu^2}{\lambda}}$. After the EWSB, the Higgs potential in the SM Lagrangian is written as follows \cite{higgs_potential}:

\begin{equation}
    V(H)= \frac{1}{2} m_H^2 H^2 + \lambda v H^3 +\frac{1}{4} \lambda H^4 + \frac{\lambda}{4} v^4
    \label{eq: higss pot after EWSB}
\end{equation}

\noindent the first term corresponds to the mass term of the Lagrangian as it contains the Higgs boson mass and the last term corresponds to the vacuum energy density. The second and third terms correspond to the three-point and four-point self-interaction terms respectively. As can be observed in Eq.(\ref{eq: higss pot after EWSB}), the self-interaction terms are proportional to $\lambda$, the Higgs boson self-coupling constant. Therefore, by experimentally measuring the Higgs self-coupling it is possible to probe the shape of the scalar potential.

In order to measure this self-coupling there are currently two complementary strategies:
\begin{itemize}
    \item Direct measurements of di-Higgs (HH), which are theoretically clean but experimentally challenging due to the low cross-section of this process.
    \item Indirect measurements in single Higgs, which needs more complicated theoretical assumptions but experimentally has more statistics as the single Higgs cross-section is around 1000 times larger than the di-Higgs production cross-section.
\end{itemize}

In the following sections, the focus will be only on the direct measurements of HH. In particular, by measuring the HH cross-section it is possible to get information on $\lambda$ as the latter can be parameterized in terms of anomalous Higgs boson couplings $\kappa_\lambda$, where $\kappa_\lambda$ is defined as follows:

\begin{equation}
    \kappa_\lambda=\frac{\lambda_{HHH}}{\lambda^{SM}_{HHH}}
\end{equation}


The dominant HH production modes in the SM are:
\begin{itemize}
    \item Gluon-gluon fusion (ggF) with a cross section $\sigma^{\text{ggF}}_{\text{HH}}=31.1^{+2.1\%}_{-7.2\%}$ fb at $\sqrt{s}=13$TeV (add Feynman diagrams?)
    \item Vector-boson fusion (VBF) with a cross section $\sigma^{\text{VBF}}_{\text{HH}}= 1.726 \pm 0.036$ fb at $\sqrt{s}=13$TeV (add Feynman diagrams?)
\end{itemize}


The branching ratios (BR) of the Higgs boson are shown in Figure \ref{fig: BR Higgs}. As can be observed in the Figure, for $m_H=125$GeV the highest BR is given by the $H\to b\Bar{b}$ channel. Nevertheless, proton-proton collisions occur at the LHC, therefore we have a high QCD background of multijet production, hence this channel's low signal/background ratio (S/B). However, even considering the low S/B of this process, since the HH cross section is so small we will be interested in the $HH \to 4b$ channel, as it has the highest BR.

\begin{figure}[hbt]
    \centering
    \includegraphics[width=0.7\linewidth]{Images/4.HH4b Analysis/Higgs intro/BR_higgs.png}
    \caption{Branching ratios and their uncertainties of the Higgs boson decay channels as a function oh the Higgs mass $M_H$ (cite)}
    \label{fig: BR Higgs}
\end{figure}

There are two main $HH \to 4b$ types of searches:
\begin{itemize}
    \item Resolved searches, where each $b$ quark is reconstructed as separate jets with radius (R=0.4)
    \item Boosted searches, where the $H\to bb$ decay is reconstructed as a single large-radius jet (R=0.8 or 1)
\end{itemize}

In the following sections, the focus will be on resolved searches for ggF HH production mechanism in the $b\Bar{b}b\Bar{b}$ decay channel.

\newpage

\subsubsection{Signature}
Figure \ref{fig: topology} shows the topology of the $HH\to 4b $ decay. As can be observed, we have 4 b quarks in our final state. These b quarks will then hadronize into B hadrons and will be detected as jets in the detector. Therefore, the signature expected in the detector is four small-radius b-jets (R=0.4)
\begin{figure}[hbt]
    \centering
    \includegraphics[width=0.5\linewidth]{Images/4.HH4b Analysis/Higgs intro/topology di-higgs.pdf}
    \caption{Topology of the $HH \to 4b$ decay}
    \label{fig: topology}
\end{figure}

In the following, the leading Higgs $H_1$ will correspond to the Higgs with the highest transverse momentum such that \pt($H_1$) > \pt($H_2$). $H_2$ will be referred to as the subleading Higgs. The leading Higgs has a mass $M_{H1}=125$ GeV while the subleading Higgs has a mass $m_{H2}=120$ GeV

\subsection{Samples} \label{subsection: samples}

The results presented in Sections \ref{section: improving} and \ref{section: s/b classification} relie on Monte Carlo (MC) simulated events (for signal and QCD multijet) as well as Run 3 2022 data. These samples are introduced in this Section. The signal events, i.e the HH production through the gluon-gluon fusion (ggHH) process are generated using \texttt{POWHEG} 2.0 and interfaced with \texttt{PYTHIA} for fragmentation and hadronization. The ggHH samples used in the following are either from:
\begin{itemize}
    \item SM official dataset, initially containing around 8M events with \kl=1
    \item \kl official dataset, initially containing around 100K events for \kl $\in \{0,2.45,5\}$
    \item \kl private datasets, initially containing 1.5M events with $\kappa_\lambda 
\in \{-2.0, -1.0, 0.0, 0.5,$ $ 1.0, 1.5, 2.0, 2.45, 3.0, 3.5, 4.0, 5.0\} $
\end{itemize}

Although in the official datasets it is possible to use samples with \kl $\in \{0,1,2.45,5\}$, the private samples are used in Section \ref{subsection: kl} as they allow to train the models for more \kl values and have significantly more statistics. Figure \ref{fig: eta h1 for validation} shows a validation study performed to ensure the validity of the private samples by comparing the distribution of observables to the distributions of the official samples. Even though Figure \ref{fig: eta h1 for validation} only shows the pseudo-rapidity $\eta$ of the leading Higgs, the validation was also performed for the following kinematic variables:

\begin{itemize}
    \item \pt, $\eta, m $ of the leading Higgs
    \item \pt, $\eta, m $ of the subleading Higgs
    \item \pt, $\eta, m, \Delta \eta_{HH}, \Delta \Phi_{HH} $ of the di-Higgs system
\end{itemize}


\begin{figure}[h!]
    \centering
    \begin{subfigure}[b]{0.4\textwidth}
        \centering
        \includegraphics[width=\textwidth]{Images/4.HH4b Analysis/Validation plots/eta 0.png}
        \caption{$\eta$ of the leading Higgs for \kl=0}
        \label{fig: kl0}
    \end{subfigure}
    \hfill
    \begin{subfigure}[b]{0.4\textwidth}
        \centering
        \includegraphics[width=\textwidth]{Images/4.HH4b Analysis/Validation plots/eta 1.png}
        \caption{$\eta$ of the leading Higgs for \kl=1}
        \label{fig: kl1}
    \end{subfigure}

    \vskip\baselineskip
    
    \begin{subfigure}[b]{0.4\textwidth}
        \centering
        \includegraphics[width=\textwidth]{Images/4.HH4b Analysis/Validation plots/eta 2p45.png}
        \caption{$\eta$ of the leading Higgs for \kl=2.45}
        \label{fig: kl2p25}
    \end{subfigure}
    \hfill
    \begin{subfigure}[b]{0.4\textwidth}
        \centering
        \includegraphics[width=\textwidth]{Images/4.HH4b Analysis/Validation plots/output.png}
        \caption{$\eta$ of the leading Higgs for \kl=5}
        \label{fig: kl5}
    \end{subfigure}
    
    \caption{Comparison of the $\eta$ of the leading Higgs for different \kl between the private and official signal samples}
    \label{fig: eta h1 for validation}
\end{figure}

For the HH to 4b analysis, the main background is QCD multijet. It is produced by \verb|MADGRAPH5_aMC@NLO| at leading order accuracy (LO) in the coupling constant $\alpha_s$." (cite) The jets from the matrix element calculations are matched to the parton shower produced by \verb|PYTHIA| using the MLM prescription." The QCD multijet background is produced per \Ht bins, Table \ref{table: QCD  multijet} reports the different \Ht bins and the corresponding cross-section.

\begin{table}[hbt]
    \centering
    \begin{tabular}{|c|c|c|}
        \hline
       Process  & \Ht bin [GeV] & $\sigma \times$ BR [pb] \\
       \hline
       QCD multijet in \Ht bins  & 200-400  & 1.968e+06 \\
        & 400 to 600  &  1.000e+05\\
        & 600 to 800 & 1.337e+04 \\
        & 800 to 1000 &  3.191e+03\\
        & 1000 to 1200  &  8.997e+02\\
        & 1200 to 1500 & 3.695e+02 \\
        & 1500 to 2000  & 1.272e+02 \\
        & from 2000 on  & 2.514e+01 \\
    \hline
    \end{tabular}
    \caption{\Ht bins produced for the QCD MC multijet background with the corresponding cross-section of the process per \Ht bin}
    \label{table: QCD  multijet}
\end{table}

Finally, data samples are used to test the background mass sculpting in Section \ref{subsection: bckg mass sculpting} and as background for the classification in Section \ref{section: s/b classification}. These datasets contain data collected in Run 2022 E, during Run 3. The dominant process in this data sample used as background is QCD multijet. Therefore, in the following, the background will always refer to the QCD multijet process. Nevertheless, a distinction will be made between MC-produced QCD samples (QCD MC) and data QCD samples (data) from Run 3.

\newpage

\subsection{Preselections for HH to 4b Run 3 analysis} \label{subsection:cutflows}
This section reports the characteristics of the Tight cuts used in Sections \ref{subsection: choice of inputs}, \ref{subsection: bckg mass sculpting}, \ref{subsection: kl}, \ref{subsection: pairing variability},  \ref{subsection: grid search} and the Loose cuts that are applied to the Run 3 data.

\begin{table}[hbt]
\centering
\begin{tabular}{|M{3cm}||M{12cm}|}
 \hline
 HLT  & \begin{verbatim}
    QuadPFJet70_50_40_35_PFBTagParticleNet_2BTagSum0p65
\end{verbatim} \\
 \hline
 Lepton Veto & \begin{itemize}
     \item e ($\mu$): \pt >15 (10) GeV,  $\eta$<2.4 
     \item e ($\mu$): \verb|mvaIso_WP80| (looseID)
     
     \item PF Iso $(\Delta R < 0.3) < 0.15$
     
     \item $d_{xy}$ barrel <0.05
     \item $d_{z}$ barrel <0.1
     \item $d_{xy}$ endcap <0.1
     \item $d_{z}$ endcap <0.2
 \end{itemize} \\
 \hline
 $\geq$ 4b jet candidates & \begin{itemize}
     \item \pt > 35 GeV
     \item $\eta$ < 2.5
     \item jetID with Lepton Veto
     \item No trigger matching
 \end{itemize} \\
 \hline
 HH reconstruction & \begin{itemize}
     \item 4 jets with Highest PNet b-tag score
     \item Jets ordered in \pt:
            \begin{itemize}
                \item \pt > [80,60,45,35]
            \end{itemize}
     \item Jets ordered in b-tag:
        \begin{itemize}
            \item Mean PNet of the leading 2 b-tag jets > 0.65
            \item 3° and 4° jet PNet b-tag > 0.2605 
        \end{itemize}
 \end{itemize}\\
 \hline
\end{tabular}
\caption{Tight cuts applied to the samples used for training. The trigger (HLT defined in Section \ref{section: CMS}) requirement is shown: 4 jets with \pt > [70,50,40,35] and the mean of the b-tag score of 2 of the jets to be above 0.65. Then the kinematic requirements of the four or more b jets required for the analysis are shown. Finally, the method to reconstruct the HH is specified.}
\label{table: Tight cuts}
\end{table}

\begin{table}[hbt]
\centering
\begin{tabular}{|M{3cm}||M{12cm}|}
 \hline
 HLT  & 
 \begin{verbatim}
    QuadPFJet70_50_40_35_PFBTagParticleNet_2BTagSum0p65
\end{verbatim}\\
 \hline
 Lepton Veto & \begin{itemize}
     \item e ($\mu$): \pt >15 (10) GeV,  $\eta$<2.4 
     \item e ($\mu$): \verb|mvaIso_WP80| (looseID)
     \item PF Iso $(\Delta R < 0.3) < 0.15$
     \item $d_{xy}$ barrel <0.05
     \item $d_{z}$ barrel <0.1
     \item $d_{xy}$ endcap <0.1
     \item $d_{z}$ endcap <0.2
 \end{itemize} \\
 \hline
 $\geq$ 4b jet candidates & \begin{itemize}
     \item \pt > 25 GeV
     \item $\eta$ < 2.5
     \item jetID with Lepton Veto
     \item No trigger matching
 \end{itemize} \\
 \hline
 HH reconstruction & \begin{itemize}
     \item All jets passing the preselections
     \item Jets ordered in \pt:
            \begin{itemize}
                \item \pt > [80,60,45,35]
            \end{itemize}
     \item Jets ordered in b-tag:
        \begin{itemize}
            \item Mean PNet of the leading 2 b-tag jets > 0.65
            \item 3° and 4° jet PNet b-tag > 0.2605 
        \end{itemize}
 \end{itemize}\\
 \hline
\end{tabular}
\caption{Loose cuts applied to the samples used for training. The trigger (HLT defined in Section \ref{section: CMS}) is shown, which requires 4 jets with \pt > [70,50,40,35] and the mean of the b-tag score of 2 of the jets to be above 0.65. Then the kinematic requirements of the four or more b jets required for the analysis are shown. Finally, the method to reconstruct the HH is specified.}
\label{table: Loose cuts}
\end{table}

Tables \ref{table: Tight cuts} and \ref{table: Loose cuts} outline the characteristics regarding the HLT and Lepton veto requirements when applying the Tight or Loose cuts respectively. They also report the preselections of the four or more jets considered for the pairing as well as the HH reconstruction requirements. The difference between Tight and Loose cuts is in the selections for the jets considered for the pairing and in the HH reconstruction. When applying Loose cuts, the jets are required to have a \pt > 25 GeV while for the Tight cuts the requirement is \pt > 35 GeV. Moreover, when applying the Tight cuts for the HH reconstruction only the 4 jets with the highest PNet b-tag score are considered whereas in the Loose cuts all the jets passing the preselections specified in Table \ref{table: Loose cuts} are considered. 

Applying the Tight cuts to the priavte SM signal sample (with \kl=1), containing around 8M events before preselections, leaves a sample containing around 1.7M events after preselections. Applying Loose cuts to this sample results in 2.1M events after preselections. This leads to an increase of 24\% in the number of events. Figure \ref{fig: validation cuts} shows the results of the study comparing the distributions of observables in the signal samples for different \kl. The distribution of the newly added events when considering Loose cuts, i.e. the events where at least one of the jets in the event has \pt $\in [25-35]$ GeV, is also shown. It is noticeable that when applying the Loose cuts the \pt and $m$ distributions are shifted to the left. This feature is understood by looking at the distribution of the newly added events, which have lower \pt and therefore shift the total distribution towards lower \pt and mass.

Finally, Tables \ref{table: SE kl 1}, \ref{table: SE kl 0}, \ref{table: SE kl 2p45} and \ref{table: SE kl 5} show the signal efficiency for \kl=\{ 1,0,2.45,5\} respectively. As expected, the signal efficiency is higher when the Loose cuts are applied.

\begin{figure}[h!]
    \centering
    \begin{subfigure}[b]{0.4\textwidth}
        \centering
        \includegraphics[width=\textwidth]{Images/4.HH4b Analysis/New events/pt h1.png}
        \caption{\pt of the leading Higgs when Loose or Tight cuts are applied to the signal sample}
        \label{fig: pt h1}
    \end{subfigure}
    \hfill
    \begin{subfigure}[b]{0.4\textwidth}
        \centering
        \includegraphics[width=\textwidth]{Images/4.HH4b Analysis/New events/pt h2.png}
        \caption{\pt of the subleading Higgs when Loose or Tight cuts are applied to the signal sample}
        \label{fig: pt h2}
    \end{subfigure}

    \vskip\baselineskip
    
    \begin{subfigure}[b]{0.4\textwidth}
        \centering
        \includegraphics[width=\textwidth]{Images/4.HH4b Analysis/New events/mass h1.png}
        \caption{Invariant mass of the leading Higgs when Loose or Tight cuts are applied to the signal sample}
        \label{fig: m h1}
    \end{subfigure}
    \hfill
    \begin{subfigure}[b]{0.4\textwidth}
        \centering
        \includegraphics[width=\textwidth]{Images/4.HH4b Analysis/New events/mass h2.png}
        \caption{Invariant mass of the subleading Higgs when Loose or Tight cuts are applied to the signal sample}
        \label{fig: m h2}
    \end{subfigure}
    
    \caption{Comparison of the distribution of two different observables (\pt and $m$) of the leading and the subleading Higgs when Tight and Loose cuts are applied. The newly added events by the Loose cuts are also shown in these figures.}
    \label{fig: validation cuts}
\end{figure}

\begin{table}[hbt]
\centering
    \begin{tabular}{ |p{10cm}|p{3cm}| }
 \hline
 \multicolumn{2}{|c|}{$\kappa_\lambda$=1} \\
 \hline
 Initial number of events & 7 391 383 \\
 Signal efficiency in the 4b region with the {Tight cuts}  & 8.77 \%\\
 Signal efficiency in the 4b region with the {Loose cuts}  &  10.51 \% \\
 \hline
 \end{tabular}
\caption{Comparison of the signal efficiency when applying Loose or Tight cuts to the official sample for \kl = 1}
\label{table: SE kl 1}
\end{table}



\begin{table}[hbt]
\centering
    \begin{tabular}{ |p{10cm}|p{3cm}| }
 \hline
 \multicolumn{2}{|c|}{Private sample for $\kappa_\lambda$=0} \\
 \hline
 Initial number of events & 1 494 015 \\
 Signal efficiency in the 4b region with the Tight cuts  & 6.95 \% \\
 Signal efficiency in the 4b region with the Loose cuts  &  8.57 \% \\
 \hline
 \end{tabular}
  \caption{Comparison of the signal efficiency when applying Loose or Tight cuts to the private sample for \kl = 0}
  \label{table: SE kl 0}
\end{table}



\begin{table}[hbt]
    \centering
     \begin{tabular}{ |p{10cm}|p{3cm}| }
 \hline
 \multicolumn{2}{|c|}{Private sample for $\kappa_\lambda$=2.45} \\
 \hline
 Initial number of events & 1 489 001 \\
 Signal efficiency in the 4b region with the Tight cuts  & 9.17 \% \\
 Signal efficiency in the 4b region with the Loose cuts  &  10.92 \% \\
 \hline
 \end{tabular}
  \caption{Comparison of the signal efficiency when applying Loose or Tight cuts to the private sample for \kl = 2.45}
  \label{table: SE kl 2p45}
\end{table}


 \begin{table}[hbt]
     \centering
      \begin{tabular}{ |p{10cm}|p{3cm}| }
         \hline
         \multicolumn{2}{|c|}{Private sample for $\kappa_\lambda$=5} \\
         \hline
         Initial number of events & 1 496 012 \\
         Signal efficiency in the 4b region with the Tight cuts  & 2.81 \% \\
         Signal efficiency in the 4b region with the Loose cuts  &  4.12 \% \\
         \hline
 \end{tabular}
 \caption{Comparison of the signal efficiency when applying Loose or Tight cuts to the private sample for \kl = 5}
 \label{table: SE kl 5}
\end{table}

\subsubsection{2b and 4b regions}

Based on the multiplicity of the b-tagging jets, the datasets can be classified into regions. The 2b region requires that the mean PNet of two of the leading b-tag jets is above 0.65 and a veto is applied to the b-tag score of the third and fourth jets. This veto requires that the PNet b-tag score of the third and fourth jets be lower than 0.2605 (close to the loose WP). The 4b corresponds to the region in which either Tight or Loose cuts are applied.

Having the definitions of 2b and 4b regions, it is possible to define:
\begin{itemize}
    \item 2b data: data sample in which the requirements of the 2b region are applied
    \item 4b data: data sample in which the requirements of the 4b region are applied
    \item 2b QCD: QCD MC sample in which the requirements of the 2b region are applied
    \item 4b QCD: QCD MC sample in which the requirements of the 4b region are applied
    \item 4b signal: the signal is defined in the 4b region, therefore we will always refer to it as signal samples
    
\end{itemize}

\clearpage


\subsection{Signal and control regions}

The signal region (SR) corresponds to the region in phase space having a large signal/background (s/b) ratio. On the contrary, the control region (CR), corresponds to a region with low s/b that is close to the SR. For this analysis, a radial distance from (125,120) GeV in the $m_{H1}-m_{H2}$ is defined as follows (cite):
\begin{equation}
    R_{HH}(m_{H1}, m_{H2})=\sqrt{(m_{H1}-125)^2+(m_{H2}-120)^2}
\end{equation}
The SR is defined such that $R_{HH} < 30$ GeV and the CR such that $30 < R_{HH} < 55$ GeV


\subsection{Background estimation} \label{subsection: bckg estimation}

To estimate the background of the signal, a data-driven approach is used in this analysis due to the inadequate modeling of the QCD-induced multijet processes. This approach uses a multidimensional kinematic reweighting technique. In this case, we want to reweight the 2b data observables to match the distributions of the 4b data observables. To do so, a neural network is trained to separate CR$_{\text{2b}}$ from CR$_{\text{4b}}$ data events. the outcome of this network is the transfer function which gives the reweight factor applied to CR$_{\text{2b}}$ data to match the distributions in the  CR$_{\text{4b}}$ region. The 2b data to which we have applied these reweighting factors is the so-called "4b-morphed data".


\subsection{Presenting the pairing run 2 analysis method}

In the detector, we reconstruct at least 4 jets coming from an event signal. To identify this event as a signal event, 4 of the reconstructed jets have to be matched to the generator-level quarks. It is important to note that there are two internal symmetries to this pairing: one regarding the exchange of the $b$ quarks coming from the Higgs, as the detector can't distinguish between quark and anti-quark, and another regarding the exchange of the Higgs, it is not important the leading-\pt Higgs is associated to $H_1$ or $H_2$ in Figure \ref{fig: topology}. Therefore when considering 4 jets for the pairing (0,1,2,3), there are 3 possible combinations for the pairing, given by A, B, and C.

\begin{equation*}
   \text{Combination A}:[(0,1),(2,3)] 
\end{equation*}
\begin{equation*}
   \text{Combination B}:[(0,2),(1,3)] 
\end{equation*}
\begin{equation*}
   \text{Combination C}:[(0,3),(1,2)] 
\end{equation*}

In the Run 2 analysis, the so-called "$D_{HH}$-method" is used for the pairing. To define the goodness of the pairing with this method a distance to the diagonal in the $m_{H1}-m_{H2}$ plane is defined:

\begin{equation}
    D_{HH}=\frac{|M_{H_1}- \kappa M_{H_2}|}{\sqrt{1+\kappa^2}}
    \label{eq: dist dhh}
\end{equation}
\noindent with $\kappa=\frac{m_{H1}}{m_{H2}}=\frac{125}{120}=1.04$, accounting for the difference between the 2 Higgs masses. If the pairing is correct, it is expected that $D_{HH}$ is closer to the diagonal as shown in Figure \textbf{bla}. This distance is computed for all the possible combinations, namely $D_{HH}^A$, $D_{HH}^B$, and $D_{HH}^C$. The algorithm to choose the most successful pairing among the possible combinations is defined as follows:
\begin{itemize}
    \item The $D_{HH}$ distances are ordered from smallest to highest: $D^1_{HH}$ < $D^2_{HH}$ < $D^3_{HH}$
    \item The difference $\Delta D_{HH}= |D_{HH}^1- D_{HH}^2|$ between the two smallest distances is computed. If
        \begin{itemize}
            \item $\Delta D_{HH} > 30 GeV$ the pairing used is the one given by the combination corresponding to $D_{HH}^1$, the latter being the smallest distance.
            \item $\Delta D_{HH} < 30 GeV$ the distance (either $D_{HH}^1$ or $D_{HH}^2$) having the Higgs with the largest \pt at the center of mass frame (\pt$(H)^*$) is chosen.
        \end{itemize}
\end{itemize}

Insert figure

Figure shows the results from the Run 2 analysis using the $D_{HH}$-method for the pairing. The observed and expected 95\% confidence limits (CL) on the total cross section $\sigma_{ggF+VBF}$ HH as a function of \kl are shown. The total cross-section is defined as the sum of the ggF and VBF production modes. From these results, the observed (expected) upper limit at the 95 \% CL of the total cross-section is set to 3.9 (7.8) times the SM prediction. Moreover, the observed (expected) value of \kl is in the range -2.3 < \kl < 9.4 (-5.0 < \kl < 12.0 ) at the 95 \% CL. (cite)

Insert figure

\subsection{Classification method for Run 3 data}

For the Run 3 data, a multivariate event classifier is used to discriminate the HH to 4b signal from the QCD multijet background. The background model used for the training of the classifier is the reweighted 2b data or 4b-morphed data that is defined in Section \ref{subsection: bckg estimation}.

\vspace{1 cm}

In the following sections, the use of the attention-based neural network SPANet is proposed to determine the best pairing of jets and as a classifier to discriminate the HH to 4b signal from the QCD multijet background. The goal of SPANet is to outperform the efficiency of the $D_{HH}$-pairing method used in Run 2 as well as the performance of the classification of the multivariate event classifier used for partially Run 3 data introduced in cite by using SPANet.


\section{SPANet and Machine Learning} \label{section: spanet architecture}





\section{Improving the pairing efficiency with SPANet}


\subsection{Choice of the optimal inputs for the training}
As mentioned in section \ref{section: spanet architecture} we can give SPANet global and sequential inputs for the training. The sequential inputs being the ones that can have an arbitrary number of vectors per event, they will correspond to the information of the jets used for the pairing. On the other hand, the global vectors are the ones that will have a single vector per event, so they correspond to event level information.

In order to find the optimal training that maximizes our pairing efficiency, we started by testing different configurations. As precised before, we want 4 jets in the final state, nevertheless due for instance to QCD processes even if in an event there is di-Higgs production decaying into 4 b quarks, we can have more than 4 jets in the final state. Hence, we want to start by checking how much do we gain in signal efficiency by considering other jets in the final state with lower b-tag score for the pairing. This can be seen in table \ref{table:signal_efficiency}. We first present the percentage of signal where the $k^{th}$ jet was matched, meaning that this $k^{th}$ jet was matched to one of the gen jets. We also show the percentage of signal of fully matched events, meaning the events for which 4 jets are matched to the 4 gen jets. Out of the total number of events 92.9\% are fully matched and the percentage of fully matched events when considering the first 4 leading jets is 90.2\%. According to table \ref{table:signal_efficiency}, we can see that most of the remaining signal efficiency is gained by considering a fifth jet for the paring (we gain 2.5\% out of 2.7\%). 

From these results, we conclude that the highest signal efficiency is obtained by considering either 4 or 5 jets for the pairing. Therefore, we will train SPANet using 4 and 5 jets as sequential inputs. We will also make one test considering 6 jets as sequential inputs, nevertheless this did not increase our performance, hence the choice of stopping at 5 jets for the pairing.


\begin{table}[h!]
\centering
\begin{tabular}{|p{1cm}||p{6cm}||p{6cm}|}
 \hline
 k  & Percentage of the signal with the $k^{th}$ jets matched & Percentage of signal fully matched events with the $k^{th}$ jet matched\\
 \hline
 4 &  ? & 90.2\\
 5 & 2.76 & 2.52 \\
 6 & 0.27 & 0.25 \\
 $\geq$7 &  $\approx$0.03 &  $\approx$0.02 \\
 \hline
\end{tabular}
\caption{Signal Efficiency}
\label{table:signal_efficiency}
\end{table}

For the first SPANet trainings, we will compare the performance of using either 4 or 5 jets as sequential inputs. We will use as first figure of merit for the performance the inclusive pairing efficiency and the \textit{total} pairing efficiency. We define the pairing efficiency as:

\begin{equation*}
    \text{Pairing efficiency}=\frac{\text{Correctly fully matched events}}{\text{Total number of fully matched events}}
\end{equation*}

However, as we explained earlier, the number of fully matched events is larger when considering a fifth jet for the pairing, hence to have a fair comparison between the trainings we need to compute the \textit{total} pairing efficiency which is given by:
\vspace{-0.2cm}
\begin{eqnarray*}
\text{Total efficiency} & = & \text{Pairing efficiency} \times \text{Fraction of fully matched events} \\
% & = & \frac{\text{Correctly fully matched events}}{\text{Total number of fully matched events}} \times \frac{\text{Total number of fully matched events}}{\text{Total number of events}} \\
& = &\frac{\text{Correctly fully matched events}}{\text{Total number of events}}     
\end{eqnarray*}

This is why in table \ref{table: First comparison} we summarize these first results by showing the configuration of the training as well as the pairing efficiency and for a fair comparison, the total pairing efficiency.

\begin{table}[h!]
\centering
\begin{tabular}{|M{2cm}|M{5cm}|M{2cm}|M{2cm}|M{2cm}|}
 \hline
 Training  & Configuration &  Pairing efficiency & Fraction of fully matched events & Total pairing efficiency \\
 \hline
 4 jets &  \raggedright 4 jets considered for the pairing and using \begin{itemize}[itemsep=0.01em]
    \item \pt
    \item $\eta$
    \item $\phi$
    \item b-tag
 \end{itemize} 
 as kinemtic information & 0.984 & 0.901 & 0.887 \\
 \hline
 4 jets 5 global & \raggedright 4 jets considered for the pairing and a 5$^{\text{th}}$ given as global information. We use \begin{itemize}[itemsep=0.01em]
    \item \pt
    \item $\eta$
    \item $\phi$
    \item b-tag
 \end{itemize} 
 as kinematic information  & 0.984 & 0.901 & 0.887\\
 \hline
  5 jets & \raggedright 5 jets considered for the pairing and using \begin{itemize}[itemsep=0.01em]
    \item \pt
    \item $\eta$
    \item $\phi$
    \item b-tag
 \end{itemize} 
 as kinematic information &  0.942 & 0.926 &  0.872 \\
 \hline
\end{tabular}
\caption{Comparison of the efficiency of the first trainings}
\label{table: First comparison}
\end{table}

As explained in the beginning, the global variables only add event level information, therefore in the training using 4 jets 5 global, the fifth jet adds information of kinematics of the system but is not used for the pairing.

From the results shown in Table \ref{table: First comparison}, we can at first say that using this configuration of the trainings, using 4 jets and 5$^{\text{th}}$ global or 4 jets as inputs are more performing. Nevertheless, by considering a 5$^{\text{th}}$ jet as either a global input or for the pairing can help the network as it has more information about the event, therefore, before stopping the trainings using 5 jets, we will try different configurations using 4 jets with a 5$^{\text{th}}$ global and 5 jets as inputs.


In Tables \ref{table:5 jets trainings} and  \ref{table:4 jets trainings}, we show the different configurations for the new trainings as well as their performance. In the new configurations we either vary the kinematical information of the jets or some of the hyperparameters of the model. In the first trainings presented earlier we used what we will call the \textit{Large model}, that uses the default SPANet hyperparameters. However, we wanted to try some of the hyperparameters shown in \textbf{cite}, and this configuration will correspond to the \textit{Lite Model}. The comparison of these models is shown in Table \ref{table:comparison_models}.

\begin{table}[h!]
    \centering
     \begin{tabular}{|c||c|c|}
      \hline
         & Large model &  Lite model\\ 
      \hline
      Hidden dimensions & 64 &  32\\ 
      \hline
      Transformer dimension& 64 & 64 \\ 
      \hline
      Embedding layers & 10 & 8 \\ 
      \hline
      Encoder layers & 6 & 6 \\ 
      \hline
      Trainable parameters & \textbf{2.2 M} & \textbf{0.5M} \\ 
      \hline
      Learning rate & 0.0015 & 0.00659 \\ 
      \hline
      Batch size & 2048 & 2048 \\ 
      \hline
      Optimizer & AdamW & AdamW \\ 
      \hline
      Number of epochs & 50 & 50 \\
      \hline
    \end{tabular}
    \caption{Comparison of Large and Lite Models}
    \label{table:comparison_models}
\end{table}


\begin{table}[h!]
\centering
\begin{tabular}{|M{2.5cm}|M{5.25cm}|M{1.75cm}|M{1.75cm}|M{1.75cm}|}
 \hline
 Training  & Configuration &  Pairing efficiency  & Fraction of fully matched events & Total pairing efficiency \\
 \hline
 5 jets & \raggedright \footnotesize \begin{itemize}[itemsep=0.001em]
    \item \pt
    \item $\eta$
    \item $\phi$
    \item b-tag
    \item Large model
 \end{itemize} & 0.942 & 0.926 & 0.872 \\
 \hline
 5 jets with b-tag preselection \pt regressed & \raggedright  \footnotesize \begin{itemize}[itemsep=0.001em]
    \item \pt
    \item $\eta$
    \item $\phi$
    \item b-tag
    \item Large model
    \item Fifth jet has a b-tag above the medium working point
 \end{itemize}  & 0.975 & 0.913 & 0.890 \\
 \hline
  5 jets \pt regressed and Lite model & \raggedright \footnotesize \begin{itemize}[itemsep=0.001em]
    \item \pt regressed
    \item $\eta$
    \item $\phi$
    \item b-tag
    \item Lite model
 \end{itemize} &  0.969 & 0.926 & 0.897\\
 \hline
 5 jets \pt regressed, Lite model and b-tag preselection & \raggedright \footnotesize \begin{itemize}[itemsep=0.001em]
    \item \pt
    \item $\eta$
    \item $\phi$
    \item b-tag
    \item Lite model
    \item Fifth jet has a b-tag above the medium working point
 \end{itemize} 
  & 0.977 & 0.913 & 0.892\\
 \hline
\end{tabular}
\caption{Different configurations for trainings with 5 jets as sequential inputs}
\label{table:5 jets trainings}
\end{table}

\begin{table}[h!]
\centering
\begin{tabular}{|M{2.5cm}|M{5.25cm}|M{1.75cm}|M{1.75cm}|M{1.75cm}|}
 \hline
 Training  & Configuration &  Pairing efficiency  & Fraction of fully matched events & Total pairing efficiency \\
 \hline
 4 jets 5 global & \raggedright \footnotesize \begin{itemize}[itemsep=0.001em]
    \item \pt
    \item $\eta$
    \item $\phi$
    \item b-tag
    \item Large model
 \end{itemize} & 0.984 & 0.901 & 0.887 \\
 \hline
 4 jets 5 global \pt regressed & \raggedright  \footnotesize \begin{itemize}[itemsep=0.001em]
    \item \pt regressed
    \item $\eta$
    \item $\phi$
    \item b-tag
    \item Large model
 \end{itemize}  & 0.985 & 0.901 & 0.887 \\
 \hline
  4 jets 5 global \pt regressed and Lite model & \raggedright \footnotesize \begin{itemize}[itemsep=0.001em]
    \item \pt regressed
    \item $\eta$
    \item $\phi$
    \item b-tag
    \item Lite model
 \end{itemize} &  0.986 & 0.901 & 0.888\\
 \hline
 4 jets 5 global Lite model & \raggedright \footnotesize \begin{itemize}[itemsep=0.001em]
    \item \pt
    \item $\eta$
    \item $\phi$
    \item b-tag
    \item Lite model
 \end{itemize}  & 0.983 & 0.901 & 0.886 \\
 \hline
 4 jets 5 global with b-tag preselection & \raggedright \footnotesize \begin{itemize}[itemsep=0.001em]
    \item \pt
    \item $\eta$
    \item $\phi$
    \item b-tag
    \item Large model
    \item Fifth jet given as global input has a b-tag above the medium working point
 \end{itemize} 
  & 0.984 & 0.901 & 0.886\\
 \hline
\end{tabular}
\caption{Different configurations for trainings with 4 jets as sequential inputs}
\label{table:4 jets trainings}
\end{table}

From Tables  \ref{table:5 jets trainings} and  \ref{table:4 jets trainings}, we can see that for both either 5 jets or 4 jets 5 global, the best performance is obtained by using \pt reg and the Lite model. However here we only compare the total pairing efficiencies inclusively, but it is also possible to plot them differential as a function of $m_{HH}$, as can bee seen in Figures \textbf{bla, bla}. In Figure \textbf{bka} we can see that using different configurations for 4 jets and 5th global does not make impact much our total pairing efficiency. On the contrary, in Figure wee can see how much difference it makes to change the configuration of the training.

The next step is to compare these two models to the performance of the $D_{HH}$-method presented in section \ref{section: HH4b}. The comparison is done for events for which we have $\Delta D_{HH} > 30$ GeV, as we only could implement this method for these events. In Figure \textbf{bla} we show how these 2 models outperform the $D_{HH}$-method. 

\vspace{0.2cm}

\noindent In conclusion, we obtained 2 models:

\begin{itemize} [itemsep=0.1em]
    \item 4 jets as sequential inputs with a 5th global jet using the \pt reg $\eta, \phi$ and btag of the jets as well as the lite model
    \item 5 jets as sequential inputs using the \pt reg $\eta, \phi$ and btag of the jets as well as the lite model
\end{itemize}

\noindent which have the highest total efficiency and that outperform the pairing efficiency used in the $D_{HH}$-method.

We also tried to asses the performance of these trainings evaluated on datasets where we only have 4 jets, but these evaluations are not correct due to the difference of phase space between the train and the test files. Moreover, we performed a test by training without the b-tag of the jet as input, nevertheless the performance was much worse than by including it in the inputs.

Even though the training using 5 jets as inputs outperforms the training using 4 jets with a 5$^{\text{th}}$ global, as a next figure of merit to choose our optimal model, it is important to verify if when applying our model in data, we observe background mass sculpting.

\newpage

\subsection{Background mass sculpting}
In addition to the pairing efficiency, another figure of merit we use in order to select the best model is the mass sculpting of the background. This is a very important feature as if we evaluate our model in a sample with mostly background events, we should not observe a fake peak of events around the signal region. 

To verify this, we will use the models we presented in the last section and evaluate them in a sample of 2b data. This dataset contains data collected in 2022 during the Run 3, and as explained in section \ref{subsection:cutflows} due to the requirements of the 2b region, we have mostly background events and the signal contribution due to mistagging of the jets is negligible. Hence, if we evaluate our model here, we should not observe a peak of events in the $m_{HH}$ plane around the signal region (defined in section \ref{section: HH4b}). Indeed, in Figure \textbf{bla}, we show the 2D mass distribution of the evaluation of our model on signal events. We can see that most of the predicted events are in the signal region, as we would expect. Contrary, in Figure \textbf{bla}, we show the 2D mass distribution of the evaluation of our model on 2b data. We can see that there is not a peak of events in the signal region, which is indeed what we expected. 


For more clarity, we can compare these distributions in the 1D plane. These are shown in Figures \textbf{bla bla} for the leading and the subleading Higgs respectively. In these plots we can compare the two models more accurately and conclude that there is less sculpting when considering the training with 5 jets, for both the leading and the subleading Higgs. Therefore, the model where we give 5 jets as inputs, with \pt ref and using the lite hyperparameters not only has the best pairing efficiency but also the mass sculpting is smaller than with the 4 jets model. Therefore, in the following sections, we will use this model.

The next question that arises when performing these tests, is that after performing several trainings with 5 jets, we observed a variability in the results, which is why, we decided to delve onto the variability of these trainings.

\subsection{Studies for the training variability}
So as to asses the variability of our model, we performed several trainings fixing the seed to randomly initialize the weights. As a first test to adress this variability issue, in addition to fixing the weights, we started by verifying if the Pytorch version used for the training had an impact on the variability. To do so, we performed 3 different trainings fixing the seeds to 0, 1 and 2 for 3 different versions of Pytorch. We concluded that the variability is independent of the version we use when we fix the seed to initialize the weights, so we decided to use the latest version (2.2.2) for our future trainings. Then, using this version we performed 26 different trainings with 26 different seeds to observe the variability. The results are shown in Figure \ref{fig: 5j variability}, where we plot the validation accuracy for each training. We can see that it clusters around 2 different values: 96.6\% and 94.5\%. Around each value we have a spread of around 0.3\%. Therefore, with these results we can indeed state that there is a large variability of our model (~2\%), with which we don't outperform the pairing efficiency of Run 2. To solve this this, we aim to modify our hyperparameters in order to stabilize our model.

\begin{figure}[hbt]
    \centering
    \includegraphics[scale=0.1]{Images/6.Improving/Variability Study/5 jets variability study.png}
    \caption{Variability on the training using 5 jets as inputs \pt reg and lite model parameters}
    \label{fig: 5j variability}
\end{figure}

With the lite model we are using the following hyperparmeters that we aim to modify:

\begin{table}[hbt]
\centering
\begin{tabular}{|p{5cm}|p{4cm}|p{5cm}|}
 \hline
 Hyperparemeters  & Lite model & Test for stabilization \\
 \hline
 Learning rate & 0.00659 & $5\times 10^{-3}$, $10^{-3}$, $5\times 10^{-4}$, $10^{-4}$ \\
 \hline
 Learning rate warmup cycles & 1 & 0\\
 \hline
  Learning rate cycles & 1 & 0\\
 \hline
 Batch size & 2048 / 1024 & 2048/1024 \\
 \hline
 Number of epochs & 50/300 & 50/300 \\
 \hline
\end{tabular}
\caption{Possible variations of the hyperparameters to stabilize the model}
\label{table: }
\end{table}

In figure \textbf{bla}, we show how the learning rate changes using these new parameters. We can see that instead of reaching the peak of the learning rate after a few epochs, it starts from the maximal value and linearly decreases. After several trainings, we concluded that the most stable configuration is given by the learning rate $10^{-4}$. As we can see in Figure \textbf{bla}, when using the new hyperparameters, taking the smallest learning rate and a batch size of 2048, we see that all of our trainings now cluster around 95.9-96.4\%, which gives a spread of 0.5\%. This variability is acceptable as even taking into account this variability we outperform the Run 2 pairing efficiency. When using a batch size of 1024, our trainings cluster around the same value but with a spread of 0.3\%. However, to use a batch size of 2048 allows us to have faster trainings, therefore, since the difference in the spread is not very significant, we decided to keep a batch size of 2048. 


Finally, we studied the difference in the efficiency when varying the number of epochs. In Figure \textbf{bla} we can see that is improved by around 0.4\% when using 300 epochs. In conclusion, after this study the most stable and efficient configuration is given by the parameters showed in Table \ref{table: stable model} .

\begin{table}[hbt]
\centering
\begin{tabular}{|c|c|}
 \hline
 Parameters  & Stable model  \\
 \hline
 Learning rate &  $10^{-4}$ \\
 \hline
 Learning rate warmup cycles &  0\\
 \hline
  Learning rate cycles & 0\\
 \hline
 Batch size & 2048 \\
 \hline
 Number of epochs & 300 \\
 \hline
\end{tabular}
\caption{Configuration for the Stable model}
\label{table: stable model}
\end{table}


We will stick to this configuration for our model in the following sections. As a last test to increase the performance of our model, we perform a grid search, thank to which we are able to tune some of our parameters. This will be presented in the following section.


\subsection{Grid search}
To increase the pairing efficiency, we performed a grid search in order to tune some of the parameters. 
To do so, we used the features of the Stable model and changed the hyperparameters presented in Table \ref{table: parameters for the grid search}. 
However, contrary to what is specified in Table \ref{table: stable model} for this grid search we will use 50 epochs given the limited resources at our disposal.
As can be seen in Table \ref{table: parameters for the grid search}, we give different possible values for the hyperparameters and then we perform several tests combining them differently.
In particular for the first 3, we need to choose between the ones specifies, whereas for the L2 Penalty we can continually (in log scale) choose a value between [1e-5 : 1e-3].
The final output of this search will be the combination of hyperparameters that give the best performance which are shown in Table \ref{table: grid search}.


\begin{table}[hbt]
   \centering
   \begin{tabular}{|c|c|}
    \hline
    Hidden dimension  &  \{32, 64, 96\}  \\
    \hline
   Number of encoder layers & \{5, 6, 7, 8\} \\
    \hline
    Number of branch embedding layers &  \{1, 2, 3, 4\}\\
    \hline
     Number of branch encoder layers & 4\\
    \hline
    Number of regression layers & 3 \\
    \hline
    Number of classification layers & 1 \\
    \hline
    Focal gamma & 0.0 \\
    \hline
    L2 Penalty & [1e-5 : 1e-3] \\
    \hline
   \end{tabular}
   \caption{Final hyperparameters determined by the grid search}
   \label{table: parameters for the grid search}
   \end{table}



\begin{table}[hbt]
\centering
\begin{tabular}{|c|c|}
 \hline
 Hidden dimension  &  96  \\
 \hline
Number of encoder layers & 5 \\
 \hline
 Number of branch embedding layers &  3\\
 \hline
  Number of branch encoder layers & 4\\
 \hline
 Number of regression layers & 3 \\
 \hline
 Number of classification layers & 1 \\
 \hline
 Focal gamma & 0.0 \\
 \hline
 L2 Penalty & $7.3\times 10^{-4}$ \\
 \hline
\end{tabular}
\caption{Final hyperparameters determined by the grid search}
\label{table: grid search}
\end{table}

By using these new hyperparameters our model has 5.1 M trainable parameters compared to 0.5 M in the Stable model. 
As can be seen in Figure \textbf{bla}, by using this new hyperparameters, with which we have a much larger model, 
we don't increase that much the performance compared to the Stable model performance. 
Hence, we decided to use the hyperparameters from the Stable Model. (takes longer)

%Do i mention other traiinngs done by Mathieu?

\newpage

\subsection{Introducing the kl}

As explained in section \ref{section: HH4b}, by measuring the di-Higgs production we are aiming to study 
the Higgs boson self coupling parameter $\lambda$. To do so, by measuring the cross-section of the di-Higgs production, 
we can obtain information about the parameter lambda $\lambda$ as we can parameterize it in terms of anomalous Higgs
boson couplings $\kappa_\lambda$, where $\kappa_\lambda$ is defined as follows:

\begin{equation}
    \kappa_\lambda=\frac{\lambda_{HHH}}{\lambda^{SM}_{HHH}}
\end{equation}

As can be seen in Figure \textbf{bla}, for different $\kappa_\lambda$ the kinematics 
of our process change, therefore it is important to train SPANet accordingly. 
To do so, we will use multiple signal samples with different values of $\kappa_\lambda 
\in \{-2.0, -1.0, 0.0, 0.5, 1.0, 1.5, 2.0, 2.45, 3.0, 3.5, 4.0, 5.0\}$

Present the kl efficiencies of the first trainings just evaluated on the diffrent 
kl (slide 9- $2024_04_18-HH4b-SPANet_pairing$)

Presen the trainings using kl (sl 11/12)

Comparison od the efficiency of these new trainings sl 15


IN the presentation

%At some point I think there was something weird and I can present some of the kinematical variables I plot compared ti the official dataset.


We also compare traiings with kl as inputs or not, and there we say ok, so they perform the same so since we can't hace that info from data we choose the trainng with no kl as input.

\subsubsection{Loose/ Tight selections}

Presen the new cuts, and the distributions for the new evebts

Efficiency plots usin gtraiings with loose and tight cuts.

And accordimg to mass sculpting much better the tight


\subsection{THE OPTIMAL CONFIG}
% jets as inputs, pt reg, lite and stable model, tight cuts, no kl as inout explicitely



Preent the traiining dataset and nb of events after selections

\section{S/B classification} \label{section: s/b classification}

% loss function weights
% Direction for comparisons:
% 		1) architecture
% 		2) input features
% 		3) sample dependency
% 		4) for background from MC weight vs no weights


Now that we have found the optimal model to maximize the pairing efficiency while 
reducing as much as possible the background mass sculpting,
we would like to use SPANet as signal/background (S/B) classifier. We will try different trainings and check if we outperform the DNN used for the Run 2 data presented in the AN (cite). In the following sections, we will compare the performance of our trainings when using different input variables, as well as the sample dependency of the background. Although, we will first delve into some technical details on how to implement the weights for the loss function to use SPANet as classifier.

%Add number of events before/ after preselections

\subsection{Weights for the computation of the loss}

Firstly, in order to use SPANet as classifier, we need to modify the weights for the computation of the loss as we will now be using in the training signal \textbf{and} background events. The sample we will use for the signal events is the same as the one in section \ref{section: improving}.  For the background, we would ideally like to work with morphed 4b data. Unfortunately, we don't have access to this background so we will be using the following ones:
\begin{itemize}
	\item 2b data
	\item 2b QCD Montecarlo (MC)
    \item 4b QCD Montecarlo (MC)
\end{itemize}
As we are now using signal{and} background events, it is important to take into account that the number of events are not the same for the different processes. To do so, we will introduce the {Event weights} and the {Class weights}, that we will explain in the following sections.

%Therefore, to avoid imbalanced data classification problems, we will weight our loss function accordingly.

\subsubsection{Event weights}

The QCD MC and signal events, as the name implies, are MC simulations. When generating these simulations we use the theoretical cross section given by the SM. Nevertheless, with enough computational power it is possible generate as many events as we want of that particular process, which does not necessarily match the number of events in our detector.
%as the luminosity is not taken into account. 
Or, if a process is very computationally expensive we can't generate enough simulations that will match the number of events in real data. This is why,  Event weights are usually introduced to rescale the number of events of our process.

In the detector, the expected number of events of a process with cross section $\sigma$ will be given by (cite?):

\begin{equation*}
    N_{\text{exp}} = \lumi \times \sigma \times \epsilon
\end{equation*}

\noindent With $N_{\text{exp}}$ being the total number of expected events in the detector, $\sigma$ the cross-section of this process, \lumi the integrated luminosity and $\epsilon$ the efficiency defined as follows (cite?):

%https://ipnp.cz/scheirich/?page_id=292

\begin{equation*}
    \epsilon= N_{\text{reco}} / N_{\text{gen}}
\end{equation*}

\noindent Where $N_{\text{reco}}$ is the number of events reconstructed in the detector that pass the event selections we defined, and $N_{\text{gen}}$ is the total number of generated events.

If we use directly the number of events from the MC production for our analysis we are implying that $N_{\text{exp}}=N_{\text{reco}}$, which is incorrect as $N_{\text{reco}}$ coming from the MC simulation is arbitrary. In order to match this value to the actual value of events we would observe in the detector, we need to multiply $N_{\text{reco}}$ by a corresponding weight. When we write the number of expected events as follows:

\begin{equation*}
    N_{\text{exp}} = \lumi \times \sigma \times \frac{N_{\text{reco}}}{N_{\text{gen}}}
\end{equation*}

\noindent the expression of the weight becomes clear:

\begin{equation*}
    \text{Event weights} = \frac{\lumi \times \sigma}{N_{\text{gen}}}
\end{equation*}

However, in practice the expression of these weights can vary if we take into consideration Next to Leading Order (NLO) corrections to our process. In this case, most MC generators produce events that already have a weight that is different from 1, called generator-level weights $w_g$.
% because of the way how the MC integration is implemented in the generator)
Hence, the formula for our Event weights per event $i$ is modified as follows:

\begin{equation*}
    \text{Event weights}^i = \lumi \times \sigma \times \frac{w_g^i}{\sum_i w_g^i}
\end{equation*}

For the signal events, the Event weights are all equal up to a sign, but for the background events these are all different.
This definition of Event weights can be then used to match the number of events MC produced and the ones observed in data. However, in the following, we want to use these weights to take into account the MC production effects in the loss. Therefore, since the luminosity is just a constant number that is the same for signal and background, and our aim is not to equalize the number of events to data but to use these weights in the computation of the loss, we will not consider it in Eq.(\ref{eq: event weights}) as it is only a linear rescaling of the value of the weight.

One last point we need to take into consideration for the definition of the Event weights we will use is that the production of the QCD background is divided into $H_T$ bins. Therefore each $H_T$ bin has the same number of events and this would give us a flat distribution of the background. But this is not physical, as in reality we have a falling distribution. To take this into account, we will rescale as a function of $H_T$ bins as follows:

\begin{equation}
	\text{Event weights}^i_j =\frac{w_g^i}{\sum_j w_g^j} \times \frac{\sigma^j}{\sum_{j\in S,B}(\sigma^j)}
 \label{eq: event weights}
\end{equation}
\noindent where $\sigma$ is the cross section of the process per $H_T$ bin $j$. 

When using Eq.(\ref{eq: event weights}),  for signal events we always consider the same process hence the $\sigma$ ratio is equal to 1:
\begin{equation*}
	\frac{\sigma_S^j}{\sum_{j\in S}(\sigma_S^j)}=1
\end{equation*}

For the background we have different values of the cross-section per $H_T$ bin as we are considering many different QCD processes in the same sample, each of them having a different $\sigma$.

\vspace{0.1 cm}

\noindent Ultimately, as an example, we give the following values for the Event weights:

\begin{itemize}
    \item $\text{Event weight for signal per event} \sim 10^{-3}$
    \item $\text{Event weight for 4b-QCD per event} \sim 10^{-5}-10^{-9}$
\end{itemize}

However, a new problem arises as for signal events the weights are much larger than for background events. To give an example, if we compute the total loss as follows :

\begin{equation}
    L_{\text{tot}}= \sum_{q \in \text{batch}} [\frac{L_q \cdot w^e_q}{\sum_q w^e_q} ] ,
\label{Eq: loss event weights}
\end{equation}

\noindent we could be incorrectly classifying our events even if we have an overall low loss. It could occur that the loss for signal events is low while for background events is high, then, we will have an overall low total loss as our signal Event weights (and the sum of them) is very high, but our events will be wrongly classified.

In conclusion, we need to introduce the Event weights to account for MC production features like NLO effects and the background shape of the QCD sample.

\subsubsection{Class weights}

To account for the imbalance of the weights, we will introduce the {Class weights} that are defined as follows (cite):

\begin{equation}
    \text{Class weights}^{S,B} = (1- \beta)/ (1- \beta^{\sum_{i,j\in S,B}\text{Event weights}^i_j})
    \label{Eq:class weights}
\end{equation}

\noindent with $\beta= 1 - (1/\sum (\text{Event weights}))$.

\vspace{0.1cm}

This formula is given in (cite) and is proven to be the best way to compute these weights. As we can see in Eq.(\ref{Eq:class weights}), it takes into account the number of events in the sample, as we are summing over all the events $i$, as well as the value of the Event weight. Therefore, whether there are more signal or background events the weights will balance out this difference in the computation of the loss as we are taking into account the value of the Event weights and the number of events per class. For instance, by using this formula, we have for the Class weights:
\begin{itemize}
    \item $\text{Class weight for signal per event} \sim 2.452 \times 10^{-4}$
    \item $\text{Class weight for 4b-QCD background per event} \sim 1.999$
\end{itemize}


\subsubsection{Total weights and computation of the total loss}

\noindent Now that we presented the Event and Class weights we can compute the Total weights that are given by:
\begin{equation*}
    \text{Total weights}^i = \text{Event weights}^i \times \text{Class weights}^i
\end{equation*}
As we can see from the values in our previous sections, when multiplying our Event and Class weights we obtain a total weight per event that is balanced. The values of the Total weights summed over all events for each configuration are given in Table \ref{table: weights outside SR}.  (The distinction between reduced statistics and full statistics will be explained in section \ref{subsection: sample dep})

\begin{table}[h!]
\centering
\begin{tabular}{|M{3cm}||M{3cm}|M{3cm}|}
 \hline
 Configuration & Total weight for signal & Total weight for background \\
 \hline
 4b-QCD & 0.53 & 0.33 \\
 \hline
 2b-QCD (reduced statistics) & 1.00 & 0.63 \\
 \hline
 2b-data (reduced statistics) & 0.05 & 0.03 \\
 \hline
 2b-data (full statistics) & 2.49 &  1.51 \\
 \hline
\end{tabular}
\caption{Sum of the total weights for the configurations presented in Table \ref{table: S/B trainings}. These results are obtained by multiplying the Event and the Class weights that are used in the computation of the loss}
\label{table: weights outside SR}
\end{table}

In Table \ref{table: weights outside SR} we show that the weights are balanced (always same order of magnitude), and in their computation, we are taking into account MC production NLO and background effects without having imbalanced weights.

\noindent In practice, the computation of the loss for the classification is then given by:

\begin{equation*}
    L=\frac{\sum_{i\in \text{batch}} w^e_i l_i}{\sum_i w^e_i}
\end{equation*}

\noindent where $w^e$ are the Event weights and $l$ is the cross-entropy loss function of Pytorch (cite). In the computation of the latter, the Class weights are taken into account. Hence, in conclusion, we have a loss function where the difference in the number of events between signal and background events as well as the MC production effects are taken into account.

\subsection{Input features}

As sequential input variables for our training, we will give the 4-vector of the 5 jets with leading b-tag. Then, we will try different configurations with the global variables. Firstly, we will give as global inputs the variables used in the DNN used for Run 2 (cite):

\begin{itemize}
    \item $\Delta R$ maximum between jets
    \item $\Delta R$ minimum between jets
    \item $H_t$ of the jets
    \item \pt, $\eta, \phi, \Delta R, \text{cos}(\theta), m$ of the leading and the subleading Higgs.
    \item \pt, $\eta, \phi, \Delta R, \text{cos}(\theta)^*, m, \Delta\eta, \Delta \phi, m$ of the di-Higgs system
\end{itemize}

\noindent In the following sections, we will be referring to the ensemble of these variables as the {DNN variables}.


\subsubsection{Probability difference variable}

In addition to the DNN variables presented before, we propose to give as global input a variable that we will refer to as {Probability difference variable} (PD). It is defined as the difference between the best and second best pairings predicted by SPANet.

When a SPANet model is evaluated on a signal event the best pairing will have a very high probability (close to 1) since in truth there is one true pairing. Therefore, the second best pairing will have a very small probability and the difference of these two will be a high value. On the contrary, for the background, the best and second best probabilities will both most likely have a probability of around 0.5 as this choice of the pairing is random due to the fact that there is no true pairing in this case. The difference of these two will then have a very small value. This variable has a big discriminant power that is very useful when using SPANet as S/B classifier. In the next part we will explain how can we obtain the best and the second best pairing probabilities.


After evaluating a SPANet model on the test file, we obtain as output the pairing with highest probability, i.e it gives us as output the pair of jets with the highest probability to be matched to the generator-level quarks coming from the decay of the leading and the sub-leading Higgs. This result is what we used in section \ref{section: improving}. Nevertheless, it is also possible to have as output the full matrix with all the probabilities of the pairings for the leading and the sub-leading Higgs, as illustrated in Figure \ref{fig: probabilities matrix}.


\begin{figure}[hbt]
    \centering
    \includegraphics[width=0.8\linewidth]{Images/7.S:B/Prob diff/probability difference.pdf}
    \caption{Matrix of probabilities of the pairing of the jets. As 5 jets are considered for the pairing, we have a 5x5 matrix. The lines and the columns correspond to the jets used for the pairing. The jets of the columns correspond to the jets matched to one of the quarks coming from the decay of one Higgs and the lines to the other quark coming from the decay. In each square we will have the value of the probability of pairing these jets together. Since it is impossible to pair the same jet together, the diagonal is 0. This is a symmetric matrix as we can't distinguish quark from anti-quark in the detector. In red, we have the highest pairing probabilities that are compatible and in orange the second highest.}
    \label{fig: probabilities matrix}
\end{figure}

As we use 5 jets for the pairing, in Figure \ref{fig: probabilities matrix} a 5x5 matrix is shown. In each square we will have the value of the probability of matching these jets in the event together. Since it is impossible to pair the same jet together, the diagonal is 0. This is a symmetric matrix as we can't distinguish quark from anti-quark in the detector. 

To find the highest pairing probabilities, what SPANet does using this matrix is to first look for the overall highest probability, i.e the highest probability in the leading or sub-leading Higgs matrices. In Figure \ref{fig: probabilities matrix}, the overall highest probability is shown in dark red. Once this one has been found, we can move to the other matrix and find the highest probability that is compatible with the first one, i.e that is not sharing the same jets. The latter is showed in light red. (In this example, we see the matching is compatible since for the dark red the jets 0 and 1 are paired while for the light red the jets 3 and 2 are paired). As a next step to compute the PD variable is to find the second best pairing in this matrix. To do so, we will start by removing the best pairings determined earlier (red crosses) but we will also be removing the symmetric one in the other matrix (black crosses) as we have a symmetry between the leading and the sub-leading Higgs. Finally, we apply the same procedure to find the second best pairing. In this second example, the highest second pairing is depicted in dark orange. We find the complementary one on the other matrix in light orange.

To use this variable in the training, we will evaluate the best SPANet model presented in section \ref{subsection: Optimal config} on the signal and the background files that we will use to train the classifier. Once the evaluation is complete, we will add the PD variable found as global input to the classifier. 
In Figures \ref{fig: 2b data PD}, \ref{fig: 2b QCD PD} and \ref{fig: 4b QCD PD} we show the PD variable for signal and background events. We observe the high discriminant power of this variable, especially in Figures \ref{fig: 2b data PD} and \ref{fig: 2b QCD PD} when considering the 2b data and 2b QCD as background in our training.

\begin{figure}[hbt]
    \centering
    \includegraphics[width=0.7\linewidth]{Images/7.S:B/Prob diff/2b data reduced.png}
    \caption{Probability difference variable distribution in arc-tan for signal and 2b data background events. We show the weighted and non weighted events. As this is a 2b data sample, the weighted and non weighted distributions are the same due to the definition of Event weights}
    \label{fig: 2b data PD}
\end{figure}

\begin{figure}
    \centering
    \includegraphics[width=0.7\linewidth]{Images/7.S:B/Prob diff/2b QCD arctan.png}
    \caption{Probability difference variable distribution in arc-tan for signal and 2b QCD background events. We show the weighted and non weighted events, using the Event weights.}
    \label{fig: 2b QCD PD}
\end{figure}


\begin{figure}[hbt]
    \centering
    \includegraphics[width=0.7\linewidth]{Images/7.S:B/Prob diff/4b QCD arctan.png}
    \caption{Probability difference variable distribution in arc-tan for signal and 4b QCD background events. We show the weighted and non weighted events}
    \label{fig: 4b QCD PD}
\end{figure}

Finally, we show in Figure \ref{fig: ROC PD} the weighted ROCs of the probability difference distributions. We observe for 2b data and 2b QCD a sudden increase of FPR at 90\% of signal efficiency. This is due to the fact that most of the background events have a highest best pairing probability of around 0.5 and the second best one has a really a value close to 0, therefore we have a peak of events at $\Delta$Probability = 0.5, which corresponds to 90\% of signal efficiency and explains this excess in FPR events.

\begin{figure}
    \centering
    \includegraphics[width=0.7\linewidth]{Images/7.S:B/Prob diff/Probability difference ROC curve.png}
    \caption{Weighted ROCs of the probability difference variables distributions shown in Figures \ref{fig: 2b data PD}, \ref{fig: 2b QCD PD} and \ref{fig: 4b QCD PD} for 2b data, 2b QCD and 4b QCD}
    \label{fig: ROC PD}
\end{figure}

\clearpage

\subsection{Sample dependency} \label{subsection: sample dep}

As mentioned earlier, we would ideally like to use 4b morphed data for our training. Nevertheless, as this is not possible yet, we will test 3 different configurations for our training. For each configuration we will be comparing the training using only DNN variables as inputs or DNN and Probability difference variables. These configurations are summarized in Table \ref{table: S/B trainings}.

\begin{table}[h!]
\centering
\begin{tabular}{|M{3cm}|M{10cm}|}
 \hline
 Configuration  & Inputs  \\
 \hline
 \multirow{2}{*}[0pt]{\raisebox{-3.4\height}{4b QCD}}  & \begin{itemize}[itemsep=0.01em]
    \item 4-vector of the 5 jets in leading b-tag
    \item DNN variables (global input)
 \end{itemize} \\ 
 \cline{2-2}
   &  \begin{itemize}[itemsep=0.01em]
    \item 4-vector of the 5 jets in leading b-tag
    \item DNN and PD variables (global input)
 \end{itemize} \\
 \hline
 \multirow{2}{*}[0pt]{\raisebox{-3.4\height}{2b QCD}}  & \begin{itemize}[itemsep=0.01em]
    \item 4-vector of the 5 jets in leading b-tag
    \item DNN variables (global input)
 \end{itemize} \\ 
 \cline{2-2}
   &  \begin{itemize}[itemsep=0.01em]
    \item 4-vector of the 5 jets in leading b-tag
    \item DNN and PD variables (global input)
 \end{itemize} \\
 \hline
  \multirow{2}{*}[0pt]{\raisebox{-3.4\height}{2b data}}  & \begin{itemize}[itemsep=0.01em]
    \item 4-vector of the 5 jets in leading b-tag
    \item DNN variables (global input)
 \end{itemize} \\ 
 \cline{2-2}
   &  \begin{itemize}[itemsep=0.01em]
    \item 4-vector of the 5 jets in leading b-tag
    \item DNN and PD variables (global input)
 \end{itemize} \\
 \hline
\end{tabular}
\caption{Configuration of the different trainings for S/B classification. We will be comparing the different configurations as well as the inputs for each configuration}
\label{table: S/B trainings}
\end{table}

So as to have a better comparison to the performance of the DNN used in Run 2 presented in the AN (cite) where they used the 4b morphed data, we will extrapolate our results to have an idea what we would obtain if we used 4b data in our training. To do so, we will extrapolate the value of the area under the curve (AUC) by using the following expression:

\begin{equation}
    \text{AUC}(\text{4b-data})= \text{AUC}(\text{2b-data}_f) \times \textcolor{BlueGreen}{\frac{\text{AUC}(\text{4b-QCD})}{\text{AUC}(\text{2b-QCD}_r)}} \times \textcolor{WildStrawberry}{\frac{\text{AUC}(\text{2b-QCD}_r)}{\text{AUC}(\text{2b-data}_r)}}
    \label{eq: extrapolation}
\end{equation} 
\noindent Since the 4b morphed data and the 2b data have the same number of events, we use the value of the AUC of the full statistics 2b data (black). In order to go from 2b to 4b data, we would like to take into account the b-tag dependency of our model. This value will be given by the first ratio (\textcolor{BlueGreen}{blue}). Nevertheless, this b-tag ratio is computed using QCD MC samples, therefore we need to take into account the difference that the model (meaning using either QCD or data samples) makes. This model dependency is given by the second ratio (\textcolor{WildStrawberry}{pink}).  Finally, on both ratios we see $r$, that stands for reduced statistics. As we aim to probe the b-tag dependency and the model dependency, we perform a random removal of events so that the effective statistics are the same. In this case, we will use the statistics of the 4b data ($\sim$ 100k events), since it is the dataset with the lowest number of events.

For the extrapolation of the ROC curve we will use Eq.(\ref{eq: extrapolation})  to compute the value of the false positive rate (FPR) instead of the AUC. Nonetheless, we will apply this formula to the true positive rate (TPR) values or this will result in a rescaling of thee 2b-data ROC. This is why, for the 4b-data extrapolated ROC we will be using the modified FPR values and the TPR values from the 2b-data full statistics training. 

Our ultimate goal will then be to perform this extrapolation. However, we first need the results of the trainings with 4b-QCD, 2b-QCD reduced statistics, 2b-data reduced statistics and 2b-data full statistics that will be presented in the following sections.


\subsection{Results on the input comparison} \label{subsection: results on the trainings}

%Compare weighted and non weighted
%also for the SPANet

In this section we will present the results of the trainings shown in Table \ref{table: S/B trainings}. In Figures \ref{fig: 4b QCD comp input}, \ref{fig: 2b QCD comp input} and \ref{fig: 2b data comp input} we show the weighted ROCs of these trainings. From these figures one can see that for 2b-QCD (Fig.\ref{fig: 2b QCD comp input}) and 2b-data (Fig.\ref{fig: 2b data comp input}) samples, adding the PD variable results in a significant improvement of our classification. For the 4b-QCD configuration (Fig. \ref{fig: 4b QCD comp input})  the performance is only slightly increased.  The difference in the improvement when adding the PD variable between 4b and 2b can be understood by looking at Figures \ref{fig: 2b data PD}, , \ref{fig: 2b QCD PD} and \ref{fig: 4b QCD PD}. Indeed, we observe that the discriminant power of the PD variable is lower when using 4b-QCD.

Nevertheless, it is important to point out that we performed several trainings with the 4b configuration and it occurred that we obtained a worse performance when adding the PD. This observed feature is very surprising and to try to understand it, in the following section, we will be interested in assessing the variability of these trainings with respect to the ROCs.

\begin{figure}[hbt]
    \centering
    \includegraphics[width=0.7\linewidth]{Images/7.S:B/Inputs/4b QCD bis.png}
    \caption{Weighted ROCs comparing the performance using different global inputs for the 4b-QCD configuration presented in Table \ref{table: S/B trainings}}
    \label{fig: 4b QCD comp input}
\end{figure}

\begin{figure}[hbt]
    \centering
    \includegraphics[width=0.7\linewidth]{Images/7.S:B/Inputs/2b QCD.png}
    \caption{Weighted ROCs comparing the performance using different global inputs for the 2b-QCD configuration presented in Table \ref{table: S/B trainings}}
    \label{fig: 2b QCD comp input}
\end{figure}

\begin{figure}[hbt]
    \centering
    \includegraphics[width=0.7\linewidth]{Images/7.S:B/Inputs/2b data.png}
    \caption{Weighted ROCs comparing the performance using different global inputs for the 2b-data configuration presented in Table \ref{table: S/B trainings}}
    \label{fig: 2b data comp input}
\end{figure}

Finally, we show the non weighted ROC for the 4b-QCD configuration. We observe in Figure \ref{fig: 4b QCD ROC no weights} that the distribution is much smoother compared to the one in Figure \ref{fig: 4b QCD comp input}. This feature can be understood by looking at the SPANet output of the classification, in Figure \ref{fig: SPANet output S/B 4b QCD}. Indeed adding the weights adds a lot of fluctuation to the distribution which are then reflected in the ROC. Nevertheless, in the following we will only show the weighted ROC curves as they allow a correct comparison to the AN results.

\begin{figure}[hbt]
    \centering
    \includegraphics[width=0.7\linewidth]{Images/7.S_B/Inputs/no weights 4b QCD.png}
    \caption{Unweighted ROCs comparing the performance using different global inputs for the 4b-QCD configuration presented in Table \ref{table: S/B trainings}}
    \label{fig: 4b QCD ROC no weights}
\end{figure}

\begin{figure}
    \centering
    \includegraphics[width=0.7\linewidth]{Images/7.S:B/Classification outputs/4b QCD dnn.png}
    \caption{SPANet output for the classification training using the 4b QCD configuration and DNN as input variables. It shows the probability assigned to the signal events and the probability assigned to the background events by our network. We can see the difference in the distribution when applying the weights}
    \label{fig: SPANet output S/B 4b QCD}
\end{figure}

\clearpage

\subsection{Assessing the variability of the trainings} \label{subsection: var of training S/B}

Unlike in section \ref{subsection: pairing variability}, here we look to asses the variability of the performance seen in the ROCs instead of in the validation accuracy. Nonetheless, as was done in section \ref{subsection: pairing variability}, we performed different trainings using the same configuration but fixing the seed to randomly initialize the weights. In Figures \ref{fig: 4b QCD v ariability}, \ref{fig: 2b QCD v ariability}, \ref{fig: 2b data v ariability} we can see the results of these trainings. A summary of the AUC values seen in these figures can be found in Table \ref{table: Spread of the trainings}. From looking at these values, one finds that the variability using 2b-QCD and 2b-data configurations is acceptable, as it is for 4b-QCD using DNN variables as inputs. Nevertheless, for 4b-QCD using DNN and PD as inputs we find a large uncertainty. This wide variability seen in Figure \ref{fig: 4b QCD DNN PD}, can then explain the surprising results we pointed out in section \ref{subsection: results on the trainings}. Therefore we can't state that our performance using 4b-QCD with DNN and PD improves our performance with respect to only DNN variables. We can conclude that they are the same within the variability. 

% ases tghe variability by looking at the AUCs (before: first thing)

\begin{figure}[hbt]
\centering
\begin{subfigure}{.5\textwidth}
  \centering
  \includegraphics[width=1.1\linewidth]{Images/7.S_B/Variability/4b QCD DNN.png}
  \caption{DNN as global input}
  \label{fig: 4b QCD DNN}
\end{subfigure}%
\begin{subfigure}{.5\textwidth}
  \centering
  \includegraphics[width=1.1\linewidth]{Images/7.S_B/Variability/4b QCD prob diff and DNN.png}
  \caption{DNN and PD as global inputs}
  \label{fig: 4b QCD DNN PD}
\end{subfigure}
\caption{4b-QCD variability of the ROCs for the different inputs presented in Table \ref{table: S/B trainings}}
\label{fig: 4b QCD v ariability}
\end{figure}

\begin{figure}[hbt]
\centering
\begin{subfigure}{.5\textwidth}
  \centering
  \includegraphics[width=1.1\linewidth]{Images/7.S_B/Variability/2b QCD DNN.png}
  \caption{DNN as global input}
  \label{fig: 2b QCD DNN}
\end{subfigure}%
\begin{subfigure}{.5\textwidth}
  \centering
  \includegraphics[width=1.1\linewidth]{Images/7.S_B/Variability/2b QCD DNN and prob diff.png}
  \caption{DNN and PD as global inputs}
  \label{fig: 2b QCD DNN PD}
\end{subfigure}
\caption{2b-QCD variability of the ROCs for the different inputs presented in Table \ref{table: S/B trainings}}
\label{fig: 2b QCD v ariability}
\end{figure}

\begin{figure}[hbt]
\centering
\begin{subfigure}{.5\textwidth}
  \centering
  \includegraphics[width=1.1\linewidth]{Images/7.S_B/Variability/2b data DNN.png}
  \caption{DNN as global input}
  \label{fig: 2b data DNN}
\end{subfigure}%
\begin{subfigure}{.5\textwidth}
  \centering
  \includegraphics[width=1.1\linewidth]{Images/7.S_B/Variability/2b data DNN and prob diff.png}
  \caption{DNN and PD as global inputs}
  \label{fig: 2b data DNN PD}
\end{subfigure}
\caption{2b-data variability of the ROCs for the different inputs presented in Table \ref{table: S/B trainings}}
\label{fig: 2b data v ariability}
\end{figure}

\begin{table}[hbt]
\centering
\begin{tabular}{|M{5cm}||M{2.5cm}|M{2.5cm}|M{2.5cm}|}
 \hline
 Configuration  & Maximum value of the AUC & Minimum value of the AUC & Variability \\
 \hline
 \hline
 4b-QCD (DNN) & 0.949 & 0.932 & $\sim$0.017 \\
 \hline
 4b-QCD (DNN and PD) & 0.946 & 0.920 & $\sim$0.026 \\
 \hline
 \hline
 2b-QCD (DNN) & 0.957 & 0.951 & $\sim$0.008\\
 \hline
 2b-QCD (DNN and PD) & 0.982 & 0.969 & $\sim$0.013 \\
 \hline
 \hline
 2b-data (DNN) & 0.958 & 0.948 & $\sim$0.01 \\
 \hline
 2b-data (DNN and PD) & 0.980 & 0.972 & $\sim$0.008  \\
 \hline
\end{tabular}
\caption{Summary of the AUC values observed in Figures \ref{fig: 4b QCD v ariability}, \ref{fig: 2b QCD v ariability}, \ref{fig: 2b data v ariability} for the variability of the training presented in Table \ref{table: S/B trainings}}
\label{table: Spread of the trainings}
\end{table}


Due to some limitations in our available resources, we could not compute the variability for 2b data full statistics, nevertheless, we observed that the value of the AUC was not affected (within the variability of the training) by the difference in statistics, as is shown in Table \ref{table: 2b-data AUC r vs f}.


\begin{table}[hbt]
\centering
\begin{tabular}{|M{7cm}|M{4cm}|}
 \hline
 Configuration  & Value of the AUC \\
 \hline
 \hline
 2b-data reduced statistics (DNN) & 0.951  \\
 \hline
 2b-data full statistics (DNN) & 0.958 \\
\hline
 \hline
 2b-data reduced statistics (DNN and PD) & 0.979 \\
 \hline
 2b-data full statistics (DNN and PD) & 0.977  \\
\hline
\end{tabular}
\caption{Results of the AUC obtained after training SPANet on 2b-data full and reduced statistics}
\label{table: 2b-data AUC r vs f}
\end{table}


\subsection{4b data extrapolation} \label{subsection:4b data extrapolation inclusivce}

In the previous sections we presented the necessary results to compute the 4b data extrapolation as was explained in section \ref{subsection: sample dep}. Since we don't have an immediate solution to reduce the variability on the 4b-QCD trainings, as a first approximation for the extrapolation we will compute the b-tag ratio in Eq.(\ref{eq: extrapolation}) (\textcolor{BlueGreen}{blue}) for the 11 seeds and see where the value from the AN stands within this variability.

In Figures \ref{fig: 4b data extrapolation DNN} and \ref{fig: 4b data extrapolation DNN PD} we present the results of the extrapolation and the comparison to the ROC in the AN. The values of the AUC reported in these plots are the extrapolated values computed using Eq.(\ref{eq: extrapolation}). We only show in these plots the best and the worst performing ROCs of our training to see with more clarity where the AN ROC stands within the variability of our training. By looking at these figures we can conclude that using the PD variable does not improve our performance. For both cases, the results from the AN are within the variability of our trainings. 

\begin{figure}[hbt]
    \centering
    \includegraphics[width=0.7\linewidth]{Images/7.S:B/Extrapolation/4b data dnn.png}
    \caption{Extrapolation to 4b data using the training with 2b data full statics, 2b data reduced statistics, 2b QCD reduced statistics and 4b-QCD with DNN variables as inputs. The FPR is computed using Eq.(\ref{eq: extrapolation}) and TPR is the one from 2b data full statistics. The AUCs showed in this figure are the extrapolated using Eq.(\ref{eq: extrapolation})}
    \label{fig: 4b data extrapolation DNN}
\end{figure}

\begin{figure}[hbt]
    \centering
    \includegraphics[width=0.7\linewidth]{Images/7.S:B/Extrapolation/4b data dnn proba.png}
    \caption{Extrapolation to 4b data using the training with 2b data full statics, 2b data reduced statistics, 2b QCD reduced statistics and 4b-QCD with DNN and PD variables as inputs. The FPR is computed using Eq.(\ref{eq: extrapolation}) and TPR is the one from 2b data full statistics. The AUCs showed in this figure are the extrapolated using Eq.(\ref{eq: extrapolation}}
    \label{fig: 4b data extrapolation DNN PD}
\end{figure}

(Do I also compare here best vs best and worse vs worse? As done in the last section for the SR)

\clearpage

\subsection{Signal region}

So far we have trained our models inclusively and we have been comparing them to the results in the AN. Nevertheless, the results in the AN have been obtained by training the DNN with samples considering only events in the SR (defined in section \ref{section: HH4b}). We will start by evaluating our current inclusive models presented in the previous section in samples with events in the SR and then trained and evaluated in the SR samples.


\subsubsection{Evaluation on SR}

Before we could perform our trainings with a SR sample, we started by evaluating our inclusive trainings on SR test samples. We started by comparing the evaluation of our training in 2b-data full statistics inclusive or 2b-data full statistics in the SR. The results are shown in Figure \ref{fig: 2b data comparison ev on SR}. We can see from this figure that for the training using DNN and PD as global variables, the performance is the same on SR or inclusively (within the variability). Nevertheless, for 2b-data using only DNN as inputs we observe a significant loss in performance.

\begin{figure}[hbt]
    \centering
    \includegraphics[width=0.7\linewidth]{Images/7.S:B/SR stats/2b-data comp.png}
    \caption{Comparison of the models trained inclusively with the 2b-data configuration evaluated inclusively and in the SR. The samples named 2b data SR are the ones containing only events in the SR}
    \label{fig: 2b data comparison ev on SR}
\end{figure}

Moreover in Figures \ref{fig: ev on SR dnn pd} and \ref{fig: ev on SR dnn pd}, we show the extrapolation to the 4b-data when evaluating on SR samples. Nevertheless, we can see that our performance worsens significantly, therefore we conclude that in order to have better performance we need to perform the trainings and evaluations on SR samples as will be shown in the next section.

\begin{figure}[hbt]
    \centering
    \includegraphics[width=0.7\linewidth]{Images/7.S:B/SR stats/4b data extrapol ev on SR dnn.png}
    \caption{Extrapolation to 4b data using the inclusive trainings with 2b data full statics, 2b data reduced statistics, 2b QCD reduced statistics and 4b-QCD with DNN variables as inputs but evaluated on SR. The FPR is computed using Eq.(\ref{eq: extrapolation}) and TPR is the one from 2b data full statistics. The AUCs showed in this figure are the extrapolated using Eq.(\ref{eq: extrapolation})}
    \label{fig: ev on SR dnn pd}
\end{figure}

\begin{figure}[hbt]
    \centering
    \includegraphics[width=0.7\linewidth]{Images/7.S:B/SR stats/4b data extrapolation dnn proba ev on sr.png}
    \caption{Extrapolation to 4b data using the inclusive trainings with 2b data full statics, 2b data reduced statistics, 2b QCD reduced statistics and 4b-QCD with DNN and PD variables as inputs but evaluated on SR. The FPR is computed using Eq.(\ref{eq: extrapolation}) and TPR is the one from 2b data full statistics. The AUCs showed in this figure are the extrapolated using Eq.(\ref{eq: extrapolation})}
    \label{fig: ev on SR dnn pd}
\end{figure}

\clearpage


\subsubsection{Training and evaluation on SR}

When performing the trainings in the SR it is important to point out that, as can be seen in Table \ref{table: SR efficiency}, the SR efficiency is very little for the background, hence, we are left with very little statistics.

\begin{table}[hbt]
\centering
\begin{tabular}{|M{3cm}||M{3.5cm}|M{3.5cm}|M{3.5cm}|}
 \hline
 Sample  & Number of events after preselections & Number of events in the SR & SR efficiency\\
 \hline
 \hline
 4b-Signal & 800k & 540k & 67\%\\
 \hline
 4b-QCD & 120k & 10k & 8\% \\
 \hline
 4b-data & 130k &  &  \\
 \hline
 \hline
 2b-Signal & 420k & & \\
 \hline
 2b-QCD & 4.4M & 220k & 5\% \\
 \hline
 2b-data & 5.9M & 470k & 8\% \\
 \hline
\end{tabular}
\caption{Signal region efficiency for the different configurations used for the SPANet trainings}
\label{table: SR efficiency}
\end{table}

We show in Figure \ref{fig: Assigned prob 4b QCD SR} the probability assigned by SPANet after performing a training with the 4b-QCD configuration using only events in the signal region, that we will refer to as the 4b-QCD-SR configuration. In this case we used the DNN and PD variables as inputs. As we can observe in this figure, when considering the weighted events, we have very significant fluctuations of the background due to the lack of statistics. This is then reflected on the ROC of this distribution as can be seen in Figure \ref{fig: ROC of the assignment proba SR}.  The same feature is observed when using only DNN as input variables.

\begin{figure}[hbt]
    \centering
    \includegraphics[width=0.7\linewidth]{Images/7.S:B/SR stats/4b stats SR class output.png}
    \caption{Probability assigned by SPANet as classifier of the signal events in the SR as well as the 4b QCD background events in the signal region. For this training we used DNN and PD variables as global inputs. Due to the lack of statistics in the background, we observe big fluctuations in the weighted distribution}
    \label{fig: Assigned prob 4b QCD SR}
\end{figure}

\begin{figure}[hbt]
    \centering
    \includegraphics[width=0.7\linewidth]{Images/7.S:B/SR stats/ROC 4b QCD SR dnn proba.png}
    \caption{ROC of the assignment probability distribution obtained after a training and evaluation on SR samples shown in Figure \ref{fig: Assigned prob 4b QCD SR}}
    \label{fig: ROC of the assignment proba SR}
\end{figure}

Since in section \ref{subsection: var of training S/B}, we observed a large variability for the 4b-QCD configuration, we decided to test the variability of the the 4b-QCD-SR configuration trainings. We show the outcome of these trainings in Figure \ref{fig: 4b QCD SR variability} and in Table \ref{table: Spread of 4b QCD SR} we summarize the results. 

\begin{figure}[hbt]
\centering
\begin{subfigure}{.5\textwidth}
  \centering
  \includegraphics[width=1.1\linewidth]{Images/7.S:B/Variability/4b QCD sr dnn.png}
  \caption{DNN as global input}
  \label{fig: 4b QCD SR DNN}
\end{subfigure}%
\begin{subfigure}{.5\textwidth}
  \centering
  \includegraphics[width=1.1\linewidth]{Images/7.S:B/Variability/4b QCD sr dnn + prob diff.png}
  \caption{DNN and PD as global inputs}
  \label{fig: 4b QCD SR DNN PD}
\end{subfigure}
\caption{4b-QCD SR variability for the different inputs presented in Table \ref{table: S/B trainings}}
\label{fig: 4b QCD SR variability}
\end{figure}

\begin{table}[hbt]
\centering
\begin{tabular}{|M{5cm}||M{2.5cm}|M{2.5cm}|M{2.5cm}|}
 \hline
 Configuration  & Maximum value of the AUC & Minimum value of the AUC & Spread \\
 \hline
 4b-QCD-SR (DNN) & 0.932 & 0.903 & $\sim$0.029 \\
 \hline
 4b-QCD-SR (DNN and PD) & 0.941 & 0.909 & $\sim$0.033 \\
 \hline
\end{tabular}
\caption{Summary of the variability of the ROC and AUC values of the training for 4b-QCD SR training}
\label{table: Spread of 4b QCD SR}
\end{table}

As can be seen in Table \ref{table: Spread of 4b QCD SR}, we have an even higher variability than the one presented in section \ref{subsection: var of training S/B}, which can be explained by the lack of statistics of the background. Therefore, for the extrapolation for 4b-data in SR, we will do the same as in section \ref{subsection: var of training S/B}, and compute the b-tag ratio in Eq.(\ref{eq: extrapolation}) for the 11 seeds and see where the value from the AN
stands within this variability. The results are shown in Figures \ref{fig: 4b QCD SR DNN PD roc} and \ref{fig: 4b QCD SR DNN roc}. IN the SR we can conclude that adding the PD variable adds more variability to the trainings as previously noticed, however, adding this variable in the SR allows us to significantly improve our performance.

\begin{figure}[hbt]
    \centering
    \includegraphics[width=0.7\linewidth]{Images/7.S:B/SR stats/4b data DNN + pb sr.png}
    \caption{4b-data in SR extrapolated ROC using the 4b-QCD-SR, 2b-data-SR full statistics, 2b-data-SR reduced statistics and 2b-QCD-SR configurations using DNN and PD as inputs. Here the AUC corresponds to the extrapolated AUC computed using Eq.(\ref{eq: extrapolation}).  Only the best and the worst performing trainings are shown in this figure to see more clearly where does the AN ROC stand within the variability of our training}
    \label{fig: 4b QCD SR DNN PD roc}
\end{figure}

\begin{figure}[hbt]
    \centering
    \includegraphics[width=0.7\linewidth]{Images/7.S:B/SR stats/4b data DNN.png}
    \caption{4b-data in SR extrapolated ROC using the 4b-QCD-SR, 2b-data-SR full statistics, 2b-data-SR reduced statistics and 2b-QCD-SR configurations using DNN as inputs. Here the AUC corresponds to the extrapolated AUC computed using Eq.(\ref{eq: extrapolation}).  Only the best and the worst performing trainings are shown in this figure to see more clearly where does the AN ROC stand within the variability of our training}
    \label{fig: 4b QCD SR DNN roc}
\end{figure}

We conclude by looking at Figures \ref{fig: highest comp} and \ref{fig: lowest comp} seeing the summarized results in Table \ref{table: highest/ lowest SR 4b data}, that by training our model using the DNN and PD as global inputs, 0.015 is the smallest gain that we get from this method, as here we only tested the variability with 11 seeds. Moreover, by using this training, our worse performance does not outperform the DNN used in Run 2, but is still better than the one using only DNN variables as global inputs. For the next steps, in order to avoid the lack of statistics and reduce the variability, we propose to oversample the 4b-QCD-SR sample used for the training.

\begin{table}[hbt]
\centering
\begin{tabular}{|M{5cm}||M{2.5cm}|M{2.5cm}|}
 \hline
 Configuration  & Maximum value of the AUC & Minimum value of the AUC \\
 \hline
 4b-data-SR (DNN) & 0.938 & 0.909  \\
 \hline
 4b-data-SR (DNN and PD) & 0.950 & 0.918 \\
 \hline
\end{tabular}
\caption{Comparison of the highest and lowest extrapolated AUC values for 4b-data using Eq.(\ref{eq: extrapolation}) to compute them. Here wee compare the difference given by the inputs in the training.}
\label{table: highest/ lowest SR 4b data}
\end{table}

\begin{figure}[hbt]
    \centering
    \includegraphics[width=0.7\linewidth]{Images/7.S:B/SR stats/HIghest AUC ROC vs Highest AUC vs AN.png}
    \caption{Comparison of the best performing 4b-data-SR extrapolated ROC when using either DNN as input variables or DNN as PD variables. The AUCs showed in this figure are the extrapolated using Eq.(\ref{eq: extrapolation}}
    \label{fig: highest comp}
\end{figure}

\begin{figure}[hbt]
    \centering
    \includegraphics[width=0.7\linewidth]{Images/7.S:B/SR stats/lowest comp 4b data.png}
    \caption{Comparison of the worst performing 4b-data-SR extrapolated ROC when using either DNN as input variables or DNN as PD variables. The AUCs showed in this figure are the extrapolated using Eq.(\ref{eq: extrapolation})}
    \label{fig: lowest comp}
\end{figure}


Compare as a function of TPR saying that we outperform up tp 0.85 signal eff. after dominated by bckg, mention again the pb of stats . oversampling will play an important role

\section{Conclusion}

This thesis introduces the usage of SPANet, an attention-based neural network, for the HH $\to$ 4b analysis. Section \ref{section: improving} presents the results of using SPANet to find the best pairing between the reconstructed jets and the generator-level quarks and Section \ref{section: s/b classification} reports the results of using SPANet as a signal/background classifier.

\vspace{0.2 cm}

Regarding the pairing efficiency, we converged on the model that showed the highest pairing efficiency, the least background sculpting, and the smallest variability. To determine this model, first, the jet multiplicity for the training was assessed. Then, different SPANet hyperparameters and input variables were tested to find the model with the highest pairing efficiency. After finding the best two performing models, the background mass sculpting was verified and one of the two was selected. Nevertheless, due to the instability of this training with respect to the validation accuracy, the SPANet architecture had to be further modified until the hyperparameters gave rise to a more stable model. Moreover, different trainings were tested using samples with different values of $\kl$ to verify the pairing efficiency as a function of \kl. It was concluded that training the model with samples containing $\kl$ samples improved the pairing efficiency as a function of \kl and $m_{HH}$. Adding \kl explicitly as a global input did not improve the performance, therefore it was decided to not include it. Finally, the impact of different kinematic selections in the analysis was tested, i.e. using Tight or Loose cuts (Section \ref{section: HH4b}) in the training samples. It was shown that using Tight cuts for the training led to a higher pairing efficiency and less background mass sculpting. Thus, in conclusion, the SPANet model that we converged on is the SPANet - \kl - Tight selection model that uses:
\begin{itemize}
    \item \pt regressed, $\eta$, $\phi$ and b-tag of the 5 jets considered for the pairing as sequential inputs
    \item No explicit \kl input as a global variable
    \item Stable model hyperparameters (shown in Table \ref{table: stable model})
\end{itemize}
\noindent Not only this model has the best pairing efficiency among all the ones tested, but it also outperforms the pairing efficiency of the $D_{HH}$-method used in the Run 2 analysis.

\vspace{0.2 cm}

When using SPANet as a S/B classifier, signal and background samples were used for the training. Therefore, the loss function needs to be modified and to do so, event and class weights were introduced in the computation. These allow to properly account for MC production effects and the difference in the number of signal and background events and thus avoid imbalance classification problems. Ideally, the background sample for the training would be 4b-morphed data (Section \ref{section: HH4b}). Nevertheless, as it is not possible to use it at the moment, different training configurations were tested using 2b-data, 4b-data, and 4b-QCD background samples (Section \ref{section: HH4b}) in order to infer the results for 4b-data using Eq.(\ref{eq: extrapolation}). For each background configuration, different inputs were tested: either the so-called "DNN" and "Probability Difference" (PD) variables, defined in Section \ref{section: s/b classification}, or uniquely DNN variables.  To assess the performance of the trainings, the ROC curves and the AUC values were compared. The variability of the different trainings was assessed as well, and a large variability was observed for the configuration using the 4b-QCD sample. This leads to variability for the inferred 4b-data results, and the results from the DNN used for Run 3 data are then compared within this observed variability. Finally, to have a better comparison to the DNN used for the Run 3 analysis, it was decided to train and evaluate the SPANet models on SR samples. It was concluded that the DNN for Run 3 is outperformed by the novel SPANet training using the DNN and the PD variables up to 85\% signal efficiency. After this 85\% signal efficiency, the probability distribution given by SPANet is dominated by the background (Figure \ref{fig: Assigned prob 4b QCD SR}), and the SR samples used have very few statistics. Thus the oversampling of the 4b-QCD-SR sample is proposed to compare the performance of the inferred 4b-data ROC curve to the ROC curve of the DNN for Run 3. Finally, as a next step, to further test the performance of SPANet for the classification, it is proposed to train the model using 4b-morphed data. 

\section{Acknowledgements}

This thesis would not have been possible without Matteo Malucchi, my basketball buddy and my best friend. Without you, I would not have been able to do this work, let alone finish it. I am beyond thankful to have had the chance to work with you, not only because you are an amazing teacher with incredible patience, but also because it is impossible to not have fun with you. Thank you for the best six months, full of knowledge and happiness. And don't you worry, I might be transferring to the Stamford branch, but I'll be back in Scranton before you know it. \textcolor{white}{This real-life version of Jim and Pam cannot be away from each other for that long.}

I would also like to express my deep gratitude to Alessandro Calandri, for his patience and guidance throughout these months. I am extremely grateful for this opportunity you gave me and everything I learned under your supervision in these past months, although I still have a lot to learn! I also wanted to express my gratitude to Mauro Donegà, for giving me this opportunity that has been of great help to me. I am deeply thankful to the ETH CMS group that has welcomed me with open arms since the first day. You are all wonderful people, and I wish you all the best. In particular, I would like to thank Giorgia Bonomelli and Sophie Rohletter for their support and motivation, you truly are amazing friends.

Finally, I would like to express my gratitude to my family. Théo et Inès, sans vous je serais certainement perdue. Je ne pourrai jamais assez vous remercier pour tout votre soutien pendant ces mois. Vraiment, quelle chance de pouvoir vous appeler ma famille. Y por último y más importante, Mama y Papa, nunca podré agradeceros suficientemente todo el apoyo que me dais. Sin vosotros nunca hubiese podido venir aquí, ni dado tantas vueltas por el mundo. Os echo de menos todos los días pero, por más lejos que estemos, siempre os siento cerca. 



% \bibliography{bibliography}
% \bibliographystyle{plain}

\newpage

\printbibliography

\end{document}
